\chapter*{Podział pracy}
\addcontentsline{toc}{chapter}{Podział pracy}

Część implementacyjna obejmowała przygotowanie architektury projektu, modułów eksperymentów i analiz oraz zestawu algorytmów. W jej ramach Marcin Połajdowicz opracował ogólną strukturę systemu, moduły odpowiedzialne za wykonywanie eksperymentów i analiz, a także implementacje algorytmów dokładnych i konstrukcyjnych. Maciej Sztramski odpowiadał za logikę oraz implementację algorytmów metaheurystycznych.

Część teoretyczna, stanowiąca zasadniczą treść pracy magisterskiej, została podzielona według rozdziałów. Maciej Sztramski przygotował rozdziały~1--4, natomiast Marcin Połajdowicz odpowiadał za rozdziały~6--9. Rozdział~5 opracowano wspólnie, dzieląc obowiązki zgodnie z zakresem prac implementacyjnych nad poszczególnymi algorytmami.

Eksperymenty obliczeniowe przeprowadził Marcin Połajdowicz. Obejmowały one testy dla schematu licencyjnego \emph{Duolingo Super}, porównania z wariantem dominowania rzymskiego, symulacje dynamiczne oraz analizę rozszerzeń modelu licencjonowania. Marcin Połajdowicz przygotował także wszystkie wykresy i tabele prezentowane w rozdziałach~6--9 (oraz odpowiadające im materiały w rozdziale~5). Rysunki ilustrujące definicje i przykłady w rozdziale~3 (m.in. dominowanie rzymskie, licencje grupowe, przykładowe zbiory dominujące) wykonał Maciej Sztramski.

Redakcję i korektę przeprowadzono krzyżowo: Marcin Połajdowicz zredagował rozdziały~1--5, a Maciej Sztramski rozdziały~6--9 wraz z Aneksem. Aneks (załączniki) przygotował w całości Marcin Połajdowicz.
