\section*{Streszczenie}

Celem niniejszej pracy magisterskiej było opracowanie grafowego modelu optymalizacji zakupu licencji oprogramowania w sieciach społecznościowych z uwzględnieniem realnych schematów cenowych platform, takich jak Duolingo Super i Spotify Premium. Problem sformułowano jako rozszerzenie klasycznego dominowania w grafach, wprowadzając warianty dominowania rzymskiego z ograniczeniami pojemności grup licencyjnych oraz zróżnicowanymi strukturami kosztów licencji indywidualnych i grupowych.

Przeprowadzono rygorystyczną analizę złożoności obliczeniowej, wykazując NP-trudność problemu poprzez redukcję z klasycznego zagadnienia zbioru dominującego. Zaimplementowano i systematycznie porównano szerokie spektrum podejść algorytmicznych: metody dokładne (programowanie całkowitoliczbowe, ILP) oraz siedem algorytmów aproksymacyjnych, w tym heurystyki konstrukcyjne (zachłanny, losowy, oparty na zbiorze dominującym) i metaheurystyki (algorytm genetyczny, przeszukiwanie tabu, symulowane wyżarzanie, algorytm mrówkowy). Badania eksperymentalne wykonano na dziesięciu rzeczywistych sieciach ego Facebooka (53--1035 węzłów) oraz trzech rodzinach grafów syntetycznych (Barabási--Albert, Erdős--Rényi, Watts--Strogatz) o rozmiarach 20--1000 węzłów.

Wyniki pokazują, że metaheurystyki osiągają jakość rozwiązań na poziomie ok. 88--90\% optimum, przy redukcji czasu obliczeń o około jeden rząd wielkości na większych instancjach względem ILP. W testach statycznych algorytm mrówkowy uzyskał najlepsze wyniki wśród metaheurystyk (koszt 0.506 vs 0.445 dla ILP przy średnim czasie 4.98\,s vs 73.83\,s). W eksperymentach dynamicznych algorytm genetyczny stanowił najlepszy kompromis jakości i efektywności (koszt 0.409 vs 0.362 dla ILP przy czasie 0.615\,s vs 1.525\,s). Przeanalizowano także scenariusz dynamiczny symulujący ewolucję sieci w czasie, w którym algorytmy wykazały dobrą zdolność adaptacji do zmian topologii. Badania objęły ponadto osiem rozszerzeń modelu, uwzględniających zróżnicowane polityki cenowe i typy planów subskrypcyjnych inspirowane rzeczywistymi ofertami platform cyfrowych.

\textbf{Słowa kluczowe:} teoria grafów, optymalizacja kombinatoryczna, dominowanie rzymskie, sieci społecznościowe, heurystyki, metaheurystyki, programowanie całkowitoliczbowe.
