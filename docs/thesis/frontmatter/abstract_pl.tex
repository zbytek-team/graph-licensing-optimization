\section*{Streszczenie}

\textbf{PLACHOLDER:} Celem niniejszej pracy magisterskiej było opracowanie grafowego modelu optymalizacji zakupu licencji oprogramowania (np. Duolingo Super, Spotify Premium) w sieciach społecznościowych. Problem ten rozpatrzono jako rozszerzenie problemu dominowania rzymskiego w grafach, uwzględniając dodatkowe ograniczenia dotyczące maksymalnego rozmiaru grup licencyjnych (od 2 do 6 osób) oraz różnic w kosztach licencji indywidualnych i grupowych.

W pracy przeprowadzono analizę złożoności obliczeniowej problemu, wykazując jego NP-trudność. Zastosowano różnorodne podejścia algorytmiczne, od metod dokładnych (programowanie matematyczne MIP) po heurystyki i metaheurystyki (algorytm zachłanny, algorytmy genetyczne, tabu search, symulowane wyżarzanie, reinforcement learning). Badania eksperymentalne przeprowadzono na rzeczywistych danych sieci społecznościowych (Ego-Facebook, Google+) oraz na danych syntetycznych (Barabási–Albert, Erdős–Rényi, Watts–Strogatz).

Otrzymane wyniki eksperymentalne wykazały, że metaheurystyki, takie jak algorytmy genetyczne oraz reinforcement learning, zapewniają najlepsze kompromisy między jakością rozwiązania a czasem działania. Analiza dynamicznej wersji problemu potwierdziła skuteczność algorytmów adaptacyjnych przy zmianach struktury sieci w czasie.

Dodatkowo przeanalizowano wpływ różnych polityk cenowych oraz dodatkowych planów subskrypcyjnych na decyzje zakupowe użytkowników. Wskazano potencjalne zastosowania modelu w różnych branżach oraz zaproponowano kierunki dalszych badań.

\textbf{Słowa kluczowe:} teoria grafów, optymalizacja kombinatoryczna, dominowanie rzymskie, sieci społecznościowe, metaheurystyki, zakup licencji, algorytmy genetyczne, reinforcement learning.
