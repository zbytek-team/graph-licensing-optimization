\section*{Streszczenie}

Celem niniejszej pracy magisterskiej było opracowanie grafowego modelu optymalizacji zakupu licencji oprogramowania (np. Duolingo Super, Spotify Premium) w sieciach społecznościowych. Problem ten sformułowano jako rozszerzenie klasycznego dominowania (w tym dominowania rzymskiego) w grafach, przy czym uwzględniono praktyczne ograniczenia dotyczące rozmiaru grup licencyjnych oraz struktury kosztów licencji indywidualnych i grupowych.

W pracy przeprowadzono analizę złożoności obliczeniowej problemu, wykazując jego NP-trudność. Zaimplementowano i porównano podejścia algorytmiczne: metody dokładne (ILP/MIP) oraz kilka metod przybliżonych i metaheurystyk (algorytm zachłanny, algorytm genetyczny, przeszukiwanie tabu, symulowane wyżarzanie, algorytm mrówkowy). Badania eksperymentalne wykonano na rzeczywistych ego-sieciach Facebooka oraz na danych syntetycznych (Barabási-Albert, Erdős-Rényi, Watts-Strogatz).

Wyniki pokazują, że metaheurystyki (w szczególności algorytm genetyczny, przeszukiwanie tabu i symulowane wyżarzanie) pozwalają osiągać rozwiązania bliskie optymalnym przy znacznym skróceniu czasu obliczeń względem ILP na większych instancjach. Dodatkowo przeanalizowano wariant dynamiczny, w którym sieć ulega zmianom w czasie, oraz wpływ polityk cenowych i dodatkowych typów planów subskrypcyjnych na optymalne decyzje.

\textbf{Słowa kluczowe:} teoria grafów, optymalizacja kombinatoryczna, dominowanie rzymskie, sieci społecznościowe, heurystyki, metaheurystyki, programowanie całkowitoliczbowe.
