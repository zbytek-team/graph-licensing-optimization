\section*{Abstract}

\textbf{PLACHOLDER:} The aim of this master's thesis was to develop a graph-based model for optimizing software license purchases (e.g., Duolingo Super, Spotify Premium) within social networks. The problem was formulated as an extension of classical Roman domination problems in graph theory, incorporating practical constraints such as group-size limitations and specific licensing costs.

This thesis analyzes the computational complexity of the problem, demonstrating its NP-hard nature. Various algorithmic approaches were implemented and compared, including exact methods (Mixed Integer Programming - MIP), heuristics, and advanced metaheuristics (genetic algorithms, Tabu Search, simulated annealing, reinforcement learning). The algorithms were evaluated experimentally on both real-world datasets (from SNAP and NetworkRepository) and synthetic graph models (Barabási–Albert, Erdős–Rényi, Watts–Strogatz).

Experimental results indicate that metaheuristic approaches, particularly genetic algorithms and reinforcement learning, effectively balance solution quality and computational efficiency. The thesis also explores dynamic scenarios in evolving social networks, demonstrating the effectiveness of adaptive algorithms in maintaining cost-efficiency.

Furthermore, the impact of varying pricing strategies and additional subscription types (e.g., Duo, Student plans) on consumer choices was investigated. The applicability of the proposed model in different industries was discussed, along with recommendations for future research directions.

\textbf{Keywords:} graph theory, combinatorial optimization, Roman domination, social networks, metaheuristics, license purchase optimization, NP-hard problems.
