\section*{Abstract}

This thesis develops a graph-based model for optimizing software license purchases (e.g., Duolingo Super, Spotify Premium) in social networks. The problem is formulated as an extension of classical domination (including Roman domination) in graphs, incorporating realistic constraints on group sizes and pricing schemes for individual and group licenses.

We analyze the computational complexity and show NP-hardness. We implement and compare exact methods (integer linear programming) with heuristic and metaheuristic approaches (greedy, genetic algorithm, tabu search, simulated annealing, ant colony optimization). Experiments are conducted on real-world Facebook ego networks and on synthetic graphs (Barabási-Albert, Erdős-Rényi, Watts-Strogatz).

Results indicate that metaheuristics (notably genetic algorithm, tabu search and simulated annealing) achieve near-optimal solutions with significantly lower runtime than ILP for larger instances. We also study a dynamic variant in which the network evolves over time and evaluate the impact of pricing policies and additional subscription types on optimal purchasing decisions.

\textbf{Keywords:} graph theory, combinatorial optimization, Roman domination, social networks, heuristics, metaheuristics, integer programming.
