\usepackage{fontspec}
\setmainfont{Arial}

% Ensure the first paragraph in each section or chapter is indented
\usepackage{indentfirst}

% Set Polish as the main language and English as a secondary language
\usepackage{polyglossia}
\setdefaultlanguage{polish}

% Page geometry settings
\usepackage{geometry}
\geometry{
    a4paper,
    top=2.5cm,
    bottom=2.5cm,
    inner=3.5cm,
    outer=2.5cm
}

% Line spacing
\usepackage{setspace}
\setstretch{1.5}

% Custom headers and footers
\usepackage{fancyhdr}
\pagestyle{fancy}
\fancyhf{}
\fancyfoot[C]{\thepage}
\fancypagestyle{plain}{
    \fancyhf{}
    \fancyfoot[C]{\thepage}
}
\renewcommand{\headrulewidth}{0pt}

% Paragraph formatting: indentation and spacing
\usepackage{parskip}
\setlength{\parindent}{1.25cm}
\setlength{\parskip}{0pt}
\raggedbottom

% Define custom colors according to the University's guidelines
\usepackage{xcolor}
\definecolor{pantone540}{HTML}{002B5C}
\definecolor{pantoneGray}{HTML}{75787B}
\definecolor{pantone1797}{HTML}{C8102E}

% Enable graphics support and define image path
\usepackage{graphicx}
\graphicspath{{assets/images/}}

% Configure captions for tables and figures
\usepackage{caption}
\captionsetup[table]{position=above}
\captionsetup[figure]{position=below}

% Enable PDF inclusion for the title page
\usepackage{pdfpages}

% Customize section and chapter titles
\usepackage{titlesec}

\titleformat{\chapter}[hang]{\bfseries\Large\MakeUppercase}{\thechapter.}{1em}{}{}
\titlespacing*{\chapter}{0pt}{*3}{*2}

\titleformat{\section}[hang]{\bfseries\itshape\large}{\thesection}{1em}{}{}
\titlespacing*{\section}{0pt}{*2}{*1.5}

\titleformat{\subsection}[hang]{\itshape\normalsize}{\thesubsection}{1em}{}{}
\titlespacing*{\subsection}{0pt}{*1.5}{*1}

% Bibliography configuration (Biber backend with ISO-numeric style)
\usepackage[backend=biber,style=iso-numeric]{biblatex}
\addbibresource{bibliography.bib}

% Polish-specific strings for bibliography
\DefineBibliographyStrings{polish}{
    bibliography = {Wykaz literatury}
}

% Math + theorem environments
\usepackage{amsmath}
\usepackage{amsthm} % provides theorem + proof environments

% Theorem-like environments (Polish names)
\newtheorem{theorem}{Twierdzenie}[chapter]
\newtheorem{lemma}{Lemat}[chapter]
\newtheorem{corollary}{Wniosek}[chapter]
\theoremstyle{definition}
\newtheorem{definition}{Definicja}[chapter]
\theoremstyle{remark}
\newtheorem{remark}{Uwaga}[chapter]

% Algorithms
\usepackage{algorithm}
\usepackage{algpseudocode}
\usepackage{float} % for [H] placement and float styling
\floatstyle{ruled}\restylefloat{algorithm}
% Polish captions and labels for algorithms
\makeatletter
\floatname{algorithm}{Algorytm}
\renewcommand{\listalgorithmname}{Spis algorytmów}
% Use Polish labels in algpseudocode. Even if some algorithms
% do not use \Require/\Ensure, keep consistent localization.
\algrenewcommand\algorithmicrequire{Dane:}
\algrenewcommand\algorithmicensure{Wynik:}
\makeatother

% Better table rules (\toprule, \midrule, \bottomrule)
\usepackage{booktabs}

% Map Unicode non-breaking hyphen (U+2011) to a safe TeX form
\usepackage{newunicodechar}
\newunicodechar{‑}{\mbox{-}}

\newcommand{\Adj}{\mathrm{Adj}}
\DeclareMathOperator{\cost}{cost}

% Practical notes box
\usepackage[most]{tcolorbox}
\tcbset{colback=white, colframe=pantoneGray!60, sharp corners}
\newtcolorbox{practical}[1][]{enhanced, breakable, title=Uwagi praktyczne, fonttitle=\bfseries, attach title to upper, colbacktitle=pantoneGray!10, #1}
