% Set up main font using Arial (requires LuaLaTeX or XeLaTeX)
\usepackage{fontspec}
% Use a widely available sans-serif as fallback (Arial may be unavailable)
\setmainfont{TeX Gyre Heros}

% Ensure the first paragraph in each section or chapter is indented
\usepackage{indentfirst}

% Set Polish as the main language and English as a secondary language
\usepackage{polyglossia}
\setdefaultlanguage{polish}

% Page geometry settings
\usepackage{geometry}
\geometry{
    a4paper,
    top=2.5cm,
    bottom=2.5cm,
    inner=3.5cm,
    outer=2.5cm
}

% Line spacing
\usepackage{setspace}
\setstretch{1.5}

% Custom headers and footers
\usepackage{fancyhdr}
\pagestyle{fancy}
\fancyhf{}
\fancyfoot[C]{\thepage}
\fancypagestyle{plain}{
    \fancyhf{}
    \fancyfoot[C]{\thepage}
}
\renewcommand{\headrulewidth}{0pt}

% Paragraph formatting: use classic indentation, no extra parskip
\setlength{\parindent}{1.25cm}
\setlength{\parskip}{0pt}
\raggedbottom

% Define custom colors according to the University's guidelines
\usepackage{xcolor}
\definecolor{pantone540}{HTML}{002B5C}
\definecolor{pantoneGray}{HTML}{75787B}
\definecolor{pantone1797}{HTML}{C8102E}

% Enable graphics support and define image path (figures live under assets/ and assets/figures/)
\usepackage{graphicx}
\graphicspath{{assets/}{assets/figures/}}

% Configure captions for tables and figures
\usepackage{caption}
\captionsetup[table]{position=above}
\captionsetup[figure]{position=below}
\usepackage{booktabs}
\usepackage{float}

% Enable PDF inclusion for the title page
\usepackage{pdfpages}

% Customize section and chapter titles
\usepackage{titlesec}

\titleformat{\chapter}[hang]{\bfseries\Large\MakeUppercase}{\thechapter.}{1em}{}{}
\titlespacing*{\chapter}{0pt}{*3}{*2}

\titleformat{\section}[hang]{\bfseries\itshape\large}{\thesection}{1em}{}{}
\titlespacing*{\section}{0pt}{*2}{*1.5}

\titleformat{\subsection}[hang]{\itshape\normalsize}{\thesubsection}{1em}{}{}
\titlespacing*{\subsection}{0pt}{*1.5}{*1}

% Bibliography configuration (Biber backend with ISO-numeric style)
\usepackage[backend=biber,style=iso-numeric]{biblatex}
\addbibresource{bibliography.bib}

% Polish-specific strings for bibliography
\DefineBibliographyStrings{polish}{
    bibliography = {Wykaz literatury}
}

\usepackage{amsmath}
\usepackage{algorithm}      % float 'algorithm'
\usepackage{algpseudocode}  % algpseudocode over algorithmicx
% Styl floatu 'algorithm': ramki u góry i u dołu, caption na górze
\floatstyle{ruled}
\restylefloat{algorithm}

% Polskie etykiety i słowa kluczowe
% Niektóre wersje 'algorithm.sty' mogą nie definiować \ALG@name — zabezpieczamy się
\makeatletter
\@ifundefined{ALG@name}{\newcommand{\ALG@name}{Algorytm}}{\renewcommand{\ALG@name}{Algorytm}}
\makeatother
\floatname{algorithm}{Algorytm}
% Lokalizacja podpisów i słów kluczowych w algorytmach
% Jeśli dostępny jest algpseudocode/algorithmicx (algrenewcommand), użyj go;
% w przeciwnym razie ustawienia dla lokalnej, minimalistycznej wersji.
\makeatletter
\@ifundefined{algrenewcommand}{%
  % Fallback dla lokalnej wersji algpseudocode.sty
  \renewcommand{\Require}{\item[\textbf{Wejście:}]~}
  \renewcommand{\Ensure}{\item[\textbf{Wyjście:}]~}
  \renewcommand{\Return}{\item[\textbf{zwróć}]~}
  % No-op środowiska Input/Output (nie są używane w treści, ale dla zgodności)
  \newenvironment{Input}{}{}
  \newenvironment{EndInput}{}{}
  \newenvironment{Output}{}{}
  \newenvironment{EndOutput}{}{}
}{%
  \algrenewcommand\algorithmicrequire{\textbf{Wejście:}}
  \algrenewcommand\algorithmicensure{\textbf{Wyjście:}}
  \algrenewcommand\algorithmicprocedure{\textbf{Procedura}}
  \algrenewcommand\algorithmicfunction{\textbf{Funkcja}}
  \algrenewcommand\algorithmicif{\textbf{Jeśli}}
  \algrenewcommand\algorithmicthen{\textbf{to}}
  \algrenewcommand\algorithmicelse{\textbf{wpp.}}
  \algrenewcommand\algorithmicfor{\textbf{Dla}}
  \algrenewcommand\algorithmicwhile{\textbf{Dopóki}}
  \algrenewcommand\algorithmicdo{\textbf{wykonuj}}
  \algrenewcommand\algorithmicreturn{\textbf{zwróć}}
  \algrenewcommand\algorithmiccomment[1]{\hfill\(\triangleright\)~##1}
  % numeracja linii mniejszą czcionką
  \algrenewcommand\alglinenumber[1]{\footnotesize ##1}
  % Bloki Input/Output
  \algblockdefx{Input}{EndInput}{\textbf{Wejście:}}{}
  \algblockdefx{Output}{EndOutput}{\textbf{Wyjście:}}{}
}
\makeatother
\usepackage{amsthm}
% Simple theorem environments (Polish captions)
\newtheorem{theorem}{Twierdzenie}

\newcommand{\Adj}{\mathrm{Adj}}
\DeclareMathOperator{\cost}{cost}
