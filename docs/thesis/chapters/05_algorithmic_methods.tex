\chapter{Metody algorytmiczne}\label{chap:algorithms}

W tym rozdziale opisano wszystkie algorytmy używane w pracy. Każdy algorytm został przedstawiony wraz z jego główną ideą, parametrami wejściowymi i wyjściowymi oraz analizą złożoności obliczeniowej. Dla każdej metody podano również jej zalety i wady.

W całym rozdziale używane są następujące oznaczenia: $G=(V,E)$ to graf nieskierowany, $N(v)$ to zbiór sąsiadów wierzchołka $v$, a $\mathcal{L}=\{\ell_1,\dots,\ell_T\}$ to zbiór dostępnych typów licencji. Każda licencja $\ell$ ma trzy parametry: $(l_\ell, u_\ell, c_\ell)$, które oznaczają minimalną pojemność grupy, maksymalną pojemność grupy i koszt licencji. Grupa licencyjna składa się z właściciela (który kupuje licencję) oraz członków grupy wybranych spośród jego sąsiadów w grafie.

\section{Przegląd metod}
Algorytmy zastosowane w pracy można podzielić na cztery grupy:
\begin{enumerate}
  \item \textbf{Metody dokładne} -- gwarantują znalezienie optymalnego rozwiązania: programowanie całkowitoliczbowe (ILP) i algorytm naiwny
  \item \textbf{Metody dla drzew} -- programowanie dynamiczne na grafach drzewiastych
  \item \textbf{Heurystyki konstrukcyjne} -- szybkie metody budujące rozwiązanie krok po kroku: algorytm zachłanny, zbiór dominujący, algorytm losowy
  \item \textbf{Metaheurystyki} -- zaawansowane metody przeszukujące przestrzeń rozwiązań: algorytm genetyczny, przeszukiwanie tabu, algorytm mrówkowy, symulowane wyżarzanie
\end{enumerate}

\paragraph{Konwencje nazewnicze} W dalszej części pracy stosujemy nazwy polskie algorytmów, a przy pierwszym użyciu podajemy w nawiasie odpowiednik angielski, np.: programowanie całkowitoliczbowe (ILP), algorytm zachłanny (Greedy), algorytm losowy (Randomized), heurystyka zbioru dominującego (Dominating Set), algorytm genetyczny (Genetic Algorithm), przeszukiwanie tabu (Tabu Search), algorytm mrówkowy (Ant Colony Optimization), symulowane wyżarzanie (Simulated Annealing).

\section{Formalizacja problemu i konwencje}\label{sec:model-formal}

\paragraph{Definicja problemu}
Problem składa się z grafu $G=(V,E)$ oraz zbioru typów licencji $\mathcal{L}=\{\ell_1,\dots,\ell_T\}$. Każda licencja $\ell$ ma trzy parametry: koszt $c_\ell>0$, minimalną pojemność $l_\ell$ i maksymalną pojemność $u_\ell$, gdzie $1\le l_\ell\le u_\ell$.

Grupa licencyjna składa się z właściciela $i$ (który kupuje licencję $\ell$) oraz członków grupy wybranych spośród sąsiadów właściciela. Oznaczając przez $N[i]$ zbiór składający się z wierzchołka $i$ i wszystkich jego sąsiadów, grupa licencyjna to podzbiór $G\subseteq N[i]$ taki, że właściciel $i$ należy do grupy ($i\in G$) oraz rozmiar grupy mieści się w dozwolonym zakresie ($l_\ell\le |G|\le u_\ell$).

Rozwiązaniem problemu jest taki zbiór grup $\mathcal{S}$, że każdy wierzchołek grafu należy do dokładnie jednej grupy.

\paragraph{Funkcja celu}
Koszt rozwiązania to suma kosztów wszystkich utworzonych grup: $\cost(\mathcal{S})=\sum\limits_{(i,\ell,G)\in\mathcal{S}} c_\ell$.

Ważna zasada: rozmiar grupy zawsze liczy właściciela, dlatego minimalna pojemność licencji wynosi co najmniej $l_\ell\ge 1$.

\paragraph{Złożoność obliczeniowa}
Problem jest NP-trudny, co oznacza, że nie istnieje algorytm wielomianowy gwarantujący znalezienie optymalnego rozwiązania (o ile P≠NP). Można to udowodnić przez redukcję z problemu minimalnego zbioru dominującego. Szczegóły tej redukcji można znaleźć w literaturze \cite{garey1979,karp1972}.

\section{Algorytmy dokładne}

\subsection{Programowanie całkowitoliczbowe (ILP)}\label{subsec:ilp}

Programowanie całkowitoliczbowe (Integer Linear Programming, ILP) to metoda dokładna, która gwarantuje znalezienie optymalnego rozwiązania. Metoda polega na sformułowaniu problemu jako zadania optymalizacji liniowej ze zmiennymi całkowitoliczbowymi.

\paragraph{Idea metody}
Algorytm minimalizuje łączny koszt wszystkich utworzonych grup licencyjnych. Używa się dwóch typów zmiennych binarnych:
\begin{itemize}
  \item $a_{i,\ell}$ -- czy wierzchołek $i$ jest właścicielem grupy z licencją $\ell$
  \item $x_{i,j,\ell}$ -- czy wierzchołek $j$ należy do grupy właściciela $i$ z licencją $\ell$
\end{itemize}

Metoda przyjmuje na wejściu graf $G=(V,E)$ i zbiór licencji $\mathcal{L}$. Na wyjściu zwraca optymalne rozwiązanie (jeśli solver zakończy się w czasie) lub najlepsze znalezione rozwiązanie w przypadku przekroczenia limitu czasu.

\paragraph{Parametry}
Jedynym parametrem jest \texttt{time\_limit} -- maksymalny czas pracy solvera CBC. Domyślnie używany jest solver CBC (Coin-or Branch and Cut).

Model ILP. Zmienne i notacja (zmienne $x$ istnieją tylko dla $j\in N[i]$):
\[
N[i] := N(i)\cup\{i\},\quad a_{i,\ell}\in\{0,1\},\quad x_{i,j,\ell}\in\{0,1\}.
\]
\begin{align}
\min\quad & \sum_{i\in V}\sum_{\ell\in\mathcal{L}} c_\ell\, a_{i,\ell} && \text{(Funkcja celu)}\\[4pt]
& \sum_{\substack{i\in V:\ j\in N[i]}}\sum_{\ell\in\mathcal{L}} x_{i,j,\ell} = 1 && \forall j\in V \quad \text{(C1) pokrycie} \label{C1}\\[2pt]
& x_{i,j,\ell} \le a_{i,\ell} && \forall i\in V,\ j\in N[i],\ \ell\in\mathcal{L} \quad \text{(C2) sprzężenie} \label{C2}\\[2pt]
& l_\ell\, a_{i,\ell} \le \sum_{j\in N[i]} x_{i,j,\ell} \le u_\ell\, a_{i,\ell} && \forall i\in V,\ \ell\in\mathcal{L} \quad \text{(C3) pojemności} \label{C3}\\[2pt]
& x_{i,i,\ell} = a_{i,\ell} && \forall i\in V,\ \ell\in\mathcal{L} \quad \text{(C4) właściciel w grupie} \label{C4}\\[2pt]
& \sum_{\ell\in\mathcal{L}} a_{i,\ell} \le 1 && \forall i\in V \quad \text{(C5) co najwyżej jedna licencja} \label{C5}
\end{align}

\noindent Ograniczenie (C2) wydaje się zbędne, ale w rzeczywistości przyspiesza działanie solvera CBC poprzez wzmocnienie relaksacji liniowej. Ograniczenie (C4) zapewnia, że właściciel zawsze należy do swojej grupy.

\paragraph{Optymalizacje}
Aby przyspieszyć działanie algorytmu, można zastosować następujące optymalizacje:
\begin{itemize}
  \item Nie tworzy się zmiennych $a_{i,\ell}$ dla wierzchołków, które nie mogą być właścicielami licencji $\ell$ (gdy liczba sąsiadów jest mniejsza niż minimalna pojemność licencji)
  \item Eliminuje się symetrię przez ograniczenie aktywacji licencji do jednego wierzchołka, gdy kilka wierzchołków ma identycznych sąsiadów
\end{itemize}

\paragraph{Złożoność}
Liczba zmiennych w modelu zależy od rozmiaru grafu i liczby typów licencji. Zmienne $a_{i,\ell}$ wynoszą $|V| \cdot T$, gdzie $|V|$ to liczba wierzchołków, a $T$ to liczba typów licencji. Zmiennych $x_{i,j,\ell}$ jest więcej -- ich liczba zależy od liczby krawędzi w grafie.

Im większy i gęstszy graf, tym więcej zmiennych i ograniczeń ma model, co znacznie wydłuża czas obliczeń.

\paragraph{Zastosowanie praktyczne}
Metoda ILP gwarantuje znalezienie najlepszego możliwego rozwiązania, ale jest wolna dla dużych grafów. Z tego powodu używa się jej głównie jako:
\begin{itemize}
  \item Punkt odniesienia do oceny jakości innych algorytmów
  \item Metoda dla małych instancji problemu (do około 100-200 wierzchołków)
  \item Narzędzie do generowania górnych ograniczeń dla heurystyk
\end{itemize}
Dla większych grafów należy ustawiać limit czasu i zadowalać się przybliżonym rozwiązaniem.


\subsection{Algorytm naiwny}

Algorytm naiwny to najprostrza metoda dokładna, która sprawdza wszystkie możliwe rozwiązania i wybiera najlepsze. Algorytm gwarantuje znalezienie optymalnego rozwiązania, ale działa tylko dla bardzo małych grafów.

\paragraph{Idea metody}
Algorytm działa w następujący sposób:
\begin{enumerate}
  \item Generuje wszystkie możliwe podziały wierzchołków grafu na grupy
  \item Dla każdego podziału sprawdza wszystkie możliwe przypisania licencji do grup
  \item Weryfikuje, czy rozwiązanie spełnia wszystkie ograniczenia problemu
  \item Oblicza koszt rozwiązania i zapamiętuje najlepsze znalezione
\end{enumerate}

\paragraph{Parametry i ograniczenia}
Algorytm ma jeden parametr: maksymalny rozmiar grafu (\texttt{max\_n=10}). Algorytm działa tylko dla grafów o maksymalnie 10 wierzchołkach, ponieważ liczba możliwych rozwiązań rośnie bardzo szybko wraz z rozmiarem grafu.

Algorytm jest używany głównie jako punkt odniesienia dla oceny jakości innych metod. Jego główne zalety to prostota i gwarancja znalezienia najlepszego rozwiązania. Wadą jest ograniczona skalowalność.


\begin{algorithm}[H]
\caption{Algorytm naiwny -- przegląd zupełny przestrzeni rozwiązań}
\label{alg:naive}
\begin{algorithmic}[1]
\Require graf \(G=(V,E)\), typy licencji \(\mathcal{L}\), opcjonalny limit \(n\le 10\)
\If{$|V|>10$} \State \Return \textit{przerwij - graf zbyt duży} \EndIf
\State wygeneruj wszystkie partycje \(\Pi\) zbioru \(V\)
\For{każdą partycję \(P=\{P_1,\dots,P_k\}\in\Pi\)}
  \State \(A\gets\emptyset\) \Comment lista dopuszczalnych przypisań dla bloków partycji
  \For{każdy blok \(P_i\) i licencję \(\ell\in\mathcal{L}\) z \(|P_i|\in[l_\ell,u_\ell]\)}
    \For{każdy \(v\in P_i\)}
      \If{$P_i\setminus\{v\}\subseteq N(v)$}
        \State dodaj krotkę \((\ell, v, P_i\setminus\{v\})\) do \(A\)
      \EndIf
    \EndFor
  \EndFor
  \State wygeneruj wszystkie kombinacje krotek z \(A\) (po jednej na każdy blok \(P_i\))
  \For{każde pełne przypisanie}
    \If{spełnia ograniczenia} \State policz koszt i zaktualizuj optimum \EndIf
  \EndFor
\EndFor
\If{nie znaleziono rozwiązań} \State przypisz wszystkim licencje indywidualne \EndIf
\State \Return najlepsze przypisanie
\end{algorithmic}
\end{algorithm}

\paragraph{Uwagi praktyczne}
Algorytm naiwny gwarantuje znalezienie rozwiązania optymalnego dla bardzo małych grafów, dzięki czemu stanowi wiarygodny punkt odniesienia do walidacji implementacji i oceny jakości metod przybliżonych. Ze względu na nadwykładniczy wzrost przestrzeni rozwiązań jest niepraktyczny już dla umiarkowanych rozmiarów instancji; zamiast tego warto wykorzystywać go jedynie do generowania wzorców porównawczych i testów jednostkowych.

\paragraph{Złożoność obliczeniowa}
Pełna przestrzeń dopuszczalnych przypisań odpowiada wszystkim rozłącznym, spójnym partycjom $V$ z grupami o rozmiarach w $[l_\ell,u_\ell]$. Górną granicę tej przestrzeni stanowi liczba Bella:
\[
B(n)=\text{liczba wszystkich partycji zbioru }V,\quad
B(n)\in\Theta\!\Bigl(\tfrac{1}{\sqrt n}\,\bigl(\tfrac{n}{\ln n}\bigr)^{\!n}\Bigr).
\]
Z powodu dodatkowych ograniczeń ($[l_\ell,u_\ell]$, spójność), liczba rozwiązań jest mniejsza od $B(n)$, ale nadal rośnie nadwykładniczo. Można zatem oszacować złożoność na:
\[
\mathcal{O}\bigl(c^n\bigr),
\]
gdzie stała $c>1$ zależy od maksymalnego stopnia wierzchołków, zakresu rozmiarów licencji i limitu $L$ odbiorców. Ze względu na tę złożoność algorytm jest praktyczny tylko dla $n\le10$ i pełni rolę algorytmu referencyjnego.

\subsection{Programowanie dynamiczne na drzewach (\texttt{tree\_dp})}\label{subsec:treedp}
Poniżej definiujemy stan, przejścia i złożoność. Rozważamy drzewo $T=(V,E)$ z korzeniem $r$ i kierujemy krawędzie od rodzica do dzieci. Dla $u\in V$ przez $\mathrm{ch}(u)$ oznaczamy dzieci i $d(u)=|\mathrm{ch}(u)|$.

\paragraph{Stan DP}
Rozważamy dwa stany dla każdego $u$:
\begin{itemize}
  \item $F(u)$ - minimalny koszt pokrycia poddrzewa $T_u$, gdy $u$ jest włączony do grupy \emph{rodzica} (nie może być właścicielem).
  \item $G(u)$ - minimalny koszt pokrycia $T_u$, gdy $u$ \emph{nie} jest włączony do grupy rodzica (może zostać właścicielem i dołączyć część dzieci).
\end{itemize}

\paragraph{Przejścia} Jeśli $u$ zostaje właścicielem z licencją $\ell$, wybieramy podzbiór dzieci $S\subseteq \mathrm{ch}(u)$ do dołączenia do grupy $u$ (pozostałe dzieci rozwiązujemy niezależnie). Warunek pojemności: $l_\ell \le 1+|S| \le u_\ell$.
\[
\textstyle\mathrm{Own}(u,\ell) \;=\; c_\ell\; +\! \sum\limits_{v\in S} F(v)\; +\! \sum\limits_{v\in \mathrm{ch}(u)\setminus S} G(v),\qquad S\subseteq\mathrm{ch}(u).
\]
Wtedy:
\[
F(u)= \sum\limits_{v\in\mathrm{ch}(u)} G(v),\qquad
G(u)= \min\limits_{\ell\in\mathcal{L}}\ \min\limits_{\substack{S\subseteq\mathrm{ch}(u) \\ l_\ell\le 1+|S|\le u_\ell}} \Bigl[ c_\ell + \sum\limits_{v\in S} F(v) + \sum\limits_{v\notin S} G(v) \Bigr].
\]
Stan korzenia: wynik końcowy to $G(r)$.

\paragraph{Złożoność} Dla każdego $u$ i licencji enumerujemy podzbiory $S\subseteq\mathrm{ch}(u)$ - koszt $\sum_u T\cdot 2^{d(u)}$, więc $\mathcal{O}\bigl(|V|\cdot T\cdot 2^{d_{\max}}\bigr)$. Dla stałego $d_{\max}$ czas jest liniowy względem $|V|\cdot T$. Pamięć: $\mathcal{O}(|V|)$.

\begin{algorithm}[H]
\caption{Programowanie dynamiczne na drzewach -- stany $F/G$ i sklejanie}
\label{alg:treedp}
\begin{algorithmic}[1]
\Require drzewo $T=(V,E)$, $\mathcal{L}$, korzeń $r$
\For{wierzchołki w porządku postorder}
  \State policz $F(u)=\sum\_{v\in \mathrm{ch}(u)} G(v)$
  \State policz $G(u)=\min\limits\_{\ell,S} \bigl[c\_\ell+\sum\_{v\in S}F(v)+\sum\_{v\notin S}G(v)\bigr]$ z warunkiem $l\_\ell\le 1+|S|\le u\_\ell$
\EndFor
\State \Return $G(r)$
\end{algorithmic}
\end{algorithm}

\paragraph{Uwagi praktyczne}
Metoda zapewnia optimum na drzewach. W zastosowaniach praktycznych bywa użyteczna także na grafach bliskich drzewom po zastosowaniu heurystyk dekompozycji (np. usunięcie krawędzi o małym znaczeniu), lecz wtedy traci się gwarancję optymalności.

\section{Heurystyki konstrukcyjne}

\subsection{Algorytm zachłanny}\label{subsec:greedy}

Algorytm zachłanny to szybka heurystyka, która buduje rozwiązanie krok po kroku, w każdym kroku wybierając lokalnie najlepszą opcję. Algorytm nie gwarantuje znalezienia optymalnego rozwiązania, ale jest bardzo szybki i daje zazwyczaj dobre wyniki.

\paragraph{Idea metody}
Algorytm działa według następującej strategii:
\begin{enumerate}
  \item Sortuje wierzchołki malejąco według liczby sąsiadów (stopnia wierzchołka)
  \item Dla każdego wierzchołka sprawdza, czy może być właścicielem grupy
  \item Wybiera typ licencji i rozmiar grupy tak, aby zminimalizować stosunek kosztu do rozmiaru grupy
  \item Dodaje członków do grupy wybierając wierzchołki o największej liczbie sąsiadów
  \item Powtarza proces dla wszystkich niepokrytych wierzchołków
\end{enumerate}

Algorytm nie ma parametrów do dostrajania -- wszystkie decyzje są podejmowane deterministycznie na podstawie struktury grafu i kosztów licencji.

\paragraph{Złożoność}
Algorytm ma złożoność czasową $O(nT + m\log n)$, gdzie $n$ to liczba wierzchołków, $T$ to liczba typów licencji, a $m$ to liczba krawędzi. Zużywa $O(n)$ pamięci.

\begin{algorithm}[H]
\caption{Zachłanny -- wybór grup o najlepszej efektywności kosztu}
\label{alg:greedy}
\begin{algorithmic}[1]
\Require graf $G=(V,E)$, rodzina licencji $\mathcal{L}$
\State $U \gets V$ \Comment zbiór wierzchołków niepokrytych
\State posortuj $V$ malejąco po $\deg(v)$; posortuj licencje malejąco po $\max$ i rosnąco po koszcie
\For{każdy $v\in V$ w tej kolejności}
  \If{$v\notin U$} \State \textit{kontynuuj} \EndIf
  \State $S \gets (N(v)\cup\{v\})\cap U$ \Comment kandydaci do grupy
  \State wybierz deterministycznie $(\ell,s)$ i $G\subseteq S$ jak w opisie, minimalizując $c_\ell/s$ przy $s\in[l_\ell,\min\{|S|,u_\ell\}]$
  \If{wybór istnieje}
    \State dodaj grupę $(v,\ell,G)$ i ustaw $U\gets U\setminus G$
  \EndIf
\EndFor
\While{$U\neq\emptyset$}
  \State wybierz $u\in U$; jeśli możliwe - utwórz najmniejszą dopuszczalną grupę, w przeciwnym razie przydziel licencję indywidualną
  \State zaktualizuj $U$
\EndWhile
\State \Return zebrane grupy
\end{algorithmic}
\end{algorithm}

\paragraph{Zastosowanie}
Algorytm zachłanny jest bardzo szybki i daje stabilne wyniki. Z tego powodu często używa się go jako:
\begin{itemize}
  \item Podstawową metodę do porównywania z innymi algorytmami
  \item Źródło rozwiązania początkowego dla bardziej zaawansowanych metod
  \item Szybką metodę dla dużych grafów, gdzie inne algorytmy są zbyt wolne
\end{itemize}
Wadą algorytmu jest to, że podejmując lokalnie najlepsze decyzje, może przegapić lepsze rozwiązania globalne.

\subsection{Heurystyka zbioru dominującego (\texttt{dominating\_set})}\label{subsec:ds}

Najpierw budujemy kosztowo efektywny zbiór dominujący $D$ (maksymalizacja pokrycia w przeliczeniu na minimalny koszt na węzeł), następnie każdemu dominatorowi przypisujemy najtańszą dopuszczalną grupę; pozostałe węzły domykamy analogicznie. Definiujemy
\[
\mathrm{coverage}(v)=|(N[v]\cap U)|,\qquad \min\_\mathrm{cpn}(v)=\min\limits_{\ell\in\mathcal{L}\atop u_\ell\ge 1}\ \frac{c_\ell}{\min\{u_\ell,\,\mathrm{coverage}(v)\}}.
\]
Wybieramy $u$ maksymalizujące $\mathrm{coverage}(u)/\min\_\mathrm{cpn}(u)$. Złożoność typowo $\mathcal{O}(n^2T)$.

\begin{algorithm}[H]
\caption{Zbiór dominujący -- heurystyka z przypisaniem grup}
\label{alg:ds}
\begin{algorithmic}[1]
\Require graf $G=(V,E)$, $\mathcal{L}$
\State $U\gets V$, $D\gets\emptyset$
\While{$U\neq\emptyset$}
  \State dla każdego $v$ policz $\mathrm{coverage}(v)=|(N(v)\cup\{v\})\cap U|$ i $\min\_\mathrm{cpn}(v)$
  \State wybierz $u$ maksymalizujące $\mathrm{coverage}(v)/\min\_\mathrm{cpn}(v)$; jeśli nie ma, weź dowolne $u\in U$
  \State $D\gets D\cup\{u\}$, $U\gets U\setminus(N(u)\cup\{u\})$
\EndWhile
\State posortuj $D$ malejąco po $\deg$
\For{każde $u\in D$ oraz dla pozostałych węzłów}
  \State $S\gets(N(u)\cup\{u\})\cap$ nieprzydzieleni
  \State wybierz najtańszą dopuszczalną grupę; w ostateczności licencję 1
\EndFor
\State \Return grupy
\end{algorithmic}
\end{algorithm}

\paragraph{Uwagi praktyczne}
Budowanie rozwiązania wokół zbioru dominującego sprzyja niskim kosztom na grafach o wyraźnych „centrach” i niewielkich średnicach (np. strukturach małoświatowych). Metoda jest szybka, lecz podatna na wybór wstępnego porządku; zwykle warto łączyć ją z domknięciem zachłannym i/lub wykorzystać jako rozgrzewkę dla metod lokalnego przeszukiwania.

\subsection{Algorytm losowy}\label{subsec:random}

Jednoprzebiegowa procedura z elementem losowym i lokalnym fallbackiem zachłannym; złożoność $\tilde O(|V|\Delta)$. Źródło losowości: generator PCG64 z ziarnem \texttt{seed} (domyślnie 42). Losujemy permutację wierzchołków i (opcjonalnie) licencję spośród dopuszczalnych proporcjonalnie do $u_\ell/c_\ell$.

\begin{algorithm}[H]
\caption{Losowy -- dobór licencji i składu grupy}
\label{alg:randomized}
\begin{algorithmic}[1]
\Require $G=(V,E)$, $\mathcal{L}$
\State $U\gets V$, $\pi\gets$ losowa permutacja $V$
\For{node w kolejności $\pi$}
  \If{$node\notin U$} \textbf{continue}\EndIf
  \State $S\gets(N(node)\cup\{node\})\cap U$
  \If{$\exists$ dopuszczalna licencja}
    \State wylosuj $\ell$ i rozmiar $s\in[l_\ell,\min\{|S|,u_\ell\}]$, dobierz losowych członków
  \Else
    \State zrób mały krok zachłanny - wybierz najtańszą dopuszczalną grupę
  \EndIf
  \State dodaj grupę, zaktualizuj $U$
\EndFor
\While{$U\neq\emptyset$} przypisz najtańszą licencję 1 i usuń węzeł z $U$ \EndWhile
\end{algorithmic}
\end{algorithm}

\paragraph{Uwagi praktyczne}
Losowa kolejność i wybór licencji zwiększają różnorodność rozwiązań, co bywa korzystne na grafach bez wyraźnej struktury. Pojedyncze uruchomienie jest bardzo szybkie; dobre rezultaty uzyskuje się uruchamiając algorytm wielokrotnie z różnymi ziarnami i wybierając najlepszy wynik.

\section{Metaheurystyki}

Metaheurystyki to zaawansowane algorytmy przeszukujące przestrzeń rozwiązań w sposób inteligentny. W przeciwieństwie do heurystyk konstrukcyjnych, które budują rozwiązanie od zera, metaheurystyki zaczynają od pewnego rozwiązania i systematycznie je poprawiają.

\paragraph{Reprezentacja rozwiązania}
Wszystkie metaheurystyki używają tego samego sposobu reprezentacji rozwiązania:
\begin{itemize}
  \item Wektor właścicieli $a$ -- dla każdego wierzchołka przechowuje informację o tym, czy jest właścicielem grupy i jakiej licencji używa
  \item Wektor przypisań $p$ -- dla każdego wierzchołka przechowuje informację o tym, do której grupy należy
\end{itemize}

\paragraph{Operacje modyfikacji rozwiązania}
Metaheurystyki poprawiają rozwiązanie stosując następujące operacje:
\begin{itemize}
  \item \texttt{change\_license} -- zmiana typu licencji używanej przez właściciela grupy
  \item \texttt{move\_member} -- przeniesienie członka z jednej grupy do drugiej
  \item \texttt{swap\_members} -- zamiana miejscami dwóch członków z różnych grup
  \item \texttt{merge\_groups}$(i,k)$ / \texttt{split\_group}$(i)$: scal dwie grupy w jedną (jeśli $\le u_\ell$) lub rozdziel na dwie dopuszczalne.
\end{itemize}

\subsection{Algorytm genetyczny (\texttt{genetic})}\label{subsec:ga}
Populacja rozwiązań z elityzmem, selekcją turniejową, krzyżowaniem (łączenie grup rodziców według efektywności i domknięcie greedy na niepokrytych) oraz mutacją w przestrzeni sąsiedztw; start z heurystyki zachłannej i opcjonalnego warm-startu \cite{holland1975,goldberg1989}.

Hiperparametry: \texttt{population\_size} (liczność populacji), \texttt{generations} (liczba pokoleń), \texttt{elite\_fraction} (udział najlepszych osobników kopiowanych bez zmian), \texttt{crossover\_rate} (prawdopodobieństwo krzyżowania wobec mutacji), \texttt{seed} (powtarzalność). Większa populacja i liczba pokoleń zwiększają szansę na lepsze rozwiązania kosztem czasu.

\begin{algorithm}[H]
\caption{Algorytm genetyczny -- elityzm, krzyżowanie i mutacja}
\label{alg:ga}
\begin{algorithmic}[1]
\Require $G=(V,E)$, $\mathcal{L}$, parametry populacji
\State zainicjalizuj populację: opcjonalny warm start, rozwiązanie greedy, reszta losowo
\For{$t\gets 1$ to $G$}
  \State $elite \gets$ top $\lceil f\cdot P\rceil$
  \While{$|New| < P$}
    \If{$\mathsf{rand}() < cr$}
      \State $child\gets$ krzyżowanie rodziców i domknięcie greedy na niepokrytych
      \If{invalid($child$)}
        \State $child\gets$ mutacja najlepszego z rodziców
      \EndIf
    \Else
      \State $child\gets$ mutacja wygranego selekcji turniejowej
    \EndIf
  \EndWhile
  \State populacja $\gets$ $elite$ $\cup$ potomstwo; aktualizuj $best$
\EndFor
\end{algorithmic}
\end{algorithm}

\paragraph{Uwagi praktyczne}
Algorytm genetyczny dobrze łączy eksplorację i eksploatację. Sprawdza się zwłaszcza, gdy dostępny jest umiarkowany budżet czasu oraz można ustawić wysokiej jakości start (np. z algorytmu zachłannego). Wymaga dostrojenia wielkości populacji i intensywności krzyżowania; zbyt niska różnorodność prowadzi do przedwczesnej zbieżności, zbyt wysoka - do wolnego postępu.

\subsection{Przeszukiwanie tabu (\texttt{tabu\_search})}\label{subsec:tabu}
Lokalne przeszukiwanie z tabu-listą i kryterium aspiracji \cite{glover1989}; w każdej iteracji wybierany jest najlepszy sąsiad spoza tabu albo poprawiający dotychczas najlepsze rozwiązanie. \emph{Ruch} to jedna z operacji sąsiedztwa zdefiniowanych wyżej, a \emph{kluczem tabu} jest para opisująca odwrócony przydział $(j, i\to k)$ albo zmiana licencji $(i,\ell\to\ell')$. Złożoność $O(I\cdot k\cdot C)$ dla \(I\) iteracji, \(k\) sąsiadów i kosztu oceny \(C\).

Hiperparametry: \texttt{max\_iterations} (liczba iteracji), \texttt{tabu\_tenure} (czas przebywania ruchu na liście tabu), \texttt{neighbors\_per\_iter} (liczba badanych sąsiadów na iterację), \texttt{seed} (powtarzalność), opcjonalnie \texttt{deadline} (twardy limit czasu) i \texttt{initial\_solution} (rozgrzewka).

\begin{algorithm}[H]
\caption{Przeszukiwanie tabu -- najlepszy sąsiad z pamięcią tabu}
\label{alg:tabu}
\begin{algorithmic}[1]
\Require $G=(V,E)$, $\mathcal{L}$, $\texttt{tabu\_tenure}$, $\texttt{max\_iterations}$
\State $current\gets$ greedy lub warm start; $best\gets current$
\State $tabu\_list\gets$ FIFO o długości \texttt{tabu\_tenure}
\For{do \texttt{max\_iterations}}
  \State $N\gets$ \texttt{generate\_neighbors}(current, k)
  \State wybierz najtańszego kandydata spoza tabu albo z aspiracją gdy poprawia $best$
  \State jeśli brak kandydata - przerwij; w przeciwnym razie zaktualizuj $current$, $best$ i $tabu\_list$
\EndFor
\State \Return $best$
\end{algorithmic}
\end{algorithm}

\paragraph{Uwagi praktyczne}
Przeszukiwanie tabu jest skuteczne na dużych instancjach, zwłaszcza gdy definiuje się bogate operatory sąsiedztwa i rozsądnie dobiera \texttt{tabu\_tenure}. Zbyt krótka lista tabu sprzyja cyklom, zbyt długa ogranicza eksplorację; warto stosować kryterium aspiracji, aby umożliwić akceptację ruchów formalnie tabu, jeżeli poprawiają najlepszy dotychczas wynik.

\subsection{Algorytm mrówkowy (\texttt{ant\_colony\_optimization})}\label{subsec:aco}
Mrówki konstruują rozwiązania sterowane feromonem \(\tau\) i heurystyką \(\eta\) \cite{dorigo1997}; po każdej iteracji następuje parowanie (parametr \texttt{evaporation}) i depozycja proporcjonalna do odwrotności kosztu najlepszego rozwiązania. Złożoność w przybliżeniu \(O(\text{iter}\times\text{ants}\times(|V|+|E|+T))\).

Hiperparametry: \texttt{alpha} (wpływ feromonu), \texttt{beta} (wpływ heurystyki), \texttt{evaporation} (tempo parowania, zwykle \(0{.}5\text{--}0{.}95\)), \texttt{q0} (prawdopodobieństwo wyboru najlepszego kroku zamiast ruletki), \texttt{num\_ants} (liczba mrówek na iterację), \texttt{max\_iterations}, \texttt{seed}; opcjonalnie \texttt{initial\_solution} do wstępnej depozycji feromonu.

\begin{algorithm}[H]
\caption{Algorytm mrówkowy -- konstrukcja rozwiązań sterowana feromonami}
\label{alg:aco}
\begin{algorithmic}[1]
\Require $G=(V,E)$, $\mathcal{L}$, parametry $\alpha,\beta,\text{evap},q_0$
\State zainicjalizuj feromony i heurystyki; $best\gets$ greedy lub warm start; zdeponuj feromon
\For{iteracje}
  \For{każda mrówka}
    \State pokryj graf wybierając właściciela i licencję metodą best lub ruletką (parametr $q0$)
    \State waliduj i aktualizuj $best$ gdy koszt spada
  \EndFor
  \State paruj feromon i zdeponuj wzdłuż grup $best$
\EndFor
\State \Return $best$
\end{algorithmic}
\end{algorithm}

\paragraph{Uwagi praktyczne}
ACO dobrze radzi sobie na grafach, gdzie lokalne wskazówki (np. stopnie wierzchołków czy koszt jednostkowy licencji) korelują z jakością globalną. Wysokie \texttt{alpha} wzmacnia eksploatację już odkrytych ścieżek, natomiast wyższe \texttt{beta} i niższe \texttt{q0} zwiększają eksplorację. Zbyt wolne parowanie może prowadzić do zastoju, zbyt szybkie - do braku konsolidacji feromonu.

\subsection{Symulowane wyżarzanie (\texttt{simulated\_annealing})}\label{subsec:sa}
Start z rozwiązania zachłannego, następnie ruchy sąsiedztwa z akceptacją pogorszeń zależną od temperatury \(T\) i schładzaniem \(T\leftarrow \alpha T\) \cite{kirkpatrick1983}. Złożoność \(O(I\cdot C)\) dla \(I\) iteracji i kosztu oceny \(C\).

Hiperparametry (domyślne): \texttt{initial\_temperature=100.0} (początkowa akceptacja pogorszeń), \texttt{cooling\_rate=0.995} (współczynnik chłodzenia \(\alpha\)), \texttt{min\_temperature=0.001} (próg zakończenia), \texttt{max\_iterations=20000}, \texttt{max\_stall=2000} (limit kolejnych iteracji bez poprawy), \texttt{seed}; opcjonalnie \texttt{initial\_solution}, \texttt{deadline}.

\begin{algorithm}[H]
\caption{Wyżarzanie symulowane -- sąsiedztwa i akceptacja pogorszeń}
\label{alg:sa}
\begin{algorithmic}[1]
\Require $G=(V,E)$, $\mathcal{L}$, parametry $T_0, T_{min}, \alpha$
\State $current\gets$ greedy lub warm start; $best\gets current$; $T\gets T_0$
\For{do \texttt{max\_iterations}}
  \If{$T<T_{min}$ lub deadline} \textbf{break} \EndIf
  \State wylosuj ruch z $\{$change\_license, move\_member, swap\_members, merge\_groups, split\_group$\}$
  \State jeśli kandydat poprawny: zaakceptuj z prawd. $\exp(-\Delta/\max(T,\varepsilon))$
  \State aktualizuj $best$, modyfikuj $T$ i licznik zastoju
\EndFor
\State \Return $best$
\end{algorithmic}
\end{algorithm}

\paragraph{Uwagi praktyczne}
Wyżarzanie symulowane jest proste w implementacji i elastyczne. Wysoka temperatura początkowa oraz wolne chłodzenie zwiększają zdolność do ucieczki z minimów lokalnych kosztem dłuższego czasu działania. W praktyce skuteczne bywa łączenie z rozwiązaniem początkowym z heurystyki zachłannej oraz ograniczanie zbioru ruchów do tych, które najszybciej obniżają koszt.

\section{Metodyka eksperymentów}

\paragraph{Ustawienia testów}
Wszystkie eksperymenty przeprowadzono za pomocą dwóch głównych skryptów: \texttt{cli/benchmark.py} dla testów statycznych i \texttt{cli/dynamic.py} dla symulacji dynamicznych. Użyto następujących ustawień:
\begin{itemize}
  \item \textbf{Maksymalny czas działania}: 60 sekund na każdy test algorytmu
  \item \textbf{Liczba powtórzeń}: każdy test uruchamiano 2 razy z różnymi losowymi ziarnami
  \item \textbf{Rozmiary grafów}: od 20 do 3000 wierzchołków (testy statyczne) lub od 40 do 10000 (testy dynamiczne)
  \item \textbf{Rodzaje grafów}: losowe, małoświatowe i bezskalowe
\end{itemize}

\paragraph{Zbierane dane}
Dla każdego uruchomienia algorytmu zapisywano następujące informacje:
\begin{itemize}
  \item \texttt{total\_cost} -- całkowity koszt znalezionego rozwiązania
  \item \texttt{time\_ms} -- czas wykonania w milisekundach
  \item \texttt{groups} -- liczba utworzonych grup licencyjnych
  \item Statystyki rozmiarów grup (średnia, mediana, 90. percentyl)
  \item \texttt{cost\_per\_node} -- koszt przypadający na jeden wierzchołek
  \item \texttt{license\_counts\_json} -- jakich typów licencji użyto
  \item \texttt{algo\_params\_json} -- parametry zastosowane w algorytmie
  \item \texttt{valid} -- czy rozwiązanie jest poprawne
\end{itemize}

\paragraph{Testowane scenariusze}
Testy przeprowadzono dla różnych konfiguracji licencji reprezentujących rzeczywiste przypadki użycia:
\begin{itemize}
  \item \texttt{duolingo\_super} -- konfiguracja wzorowana na licencjach typu Duolingo Super
  \item \texttt{roman\_domination} -- konfiguracja oparta na problemie dominacji rzymskiej
  \item Warianty z różnymi parametrami cenowymi ($p \in \{1.5, 2.5, 3.0\}$ dla dominacji rzymskiej, $p \in \{2.0, 3.0\}$ dla Duolingo)
\end{itemize}

\paragraph{Replikowalność wyników}
Wszystkie wyniki zostały zapisane w plikach CSV zawierających pełne informacje o parametrach eksperymentów i użytych ziarnach generatorów liczb losowych. Pliki przechowywane są w ustrukturyzowanych katalogach z unikatowymi identyfikatorami zawierającymi znaczniki czasu, co umożliwia pełne odtworzenie wyników.

\section{Implementacja i architektura rozwiązania}

Poniżej przedstawiono skrócony schemat przepływu danych i komponentów implementacji. Diagram ma charakter poglądowy i odpowiada strukturze repozytorium (moduły \texttt{glopt} i skrypty analizy).

\begin{figure}[H]
  \centering
  \fbox{\parbox{0.92\linewidth}{
  \textbf{Generator grafów} (syntetyczne/realne) $\Rightarrow$ \textbf{Solver} (ILP, heurystyki, metaheurystyki) $\Rightarrow$ \\
  \textbf{Walidator rozwiązań} (spójność, pojemności) $\Rightarrow$ \textbf{Zapis CSV} (metryki) $\Rightarrow$ \\
  \textbf{Skrypty analizy} (agregaty, wykresy, figury do pracy)
  }}
  \caption{Architektura wysokopoziomowa: generator → solver → walidator → CSV → analiza.}
  \label{fig:arch_overview}
\end{figure}

Najważniejsze elementy:
\begin{itemize}
  \item \textbf{Generator grafów} — tworzy instancje na podstawie parametrów (Erdős–Rényi, Watts–Strogatz, Barabási–Albert) lub ładuje sieci rzeczywiste.
  \item \textbf{Solver} — realizuje wybraną metodę (ILP, zachłanny, losowy, dominujący, GA, SA, Tabu, ACO) z opcjonalnym \emph{warm start} z rozwiązania zachłannego.
  \item \textbf{Walidator} — sprawdza pokrycie, pojemności licencji i spójność grup (odrzuca rozwiązania niepoprawne).
  \item \textbf{Zapis wyników} — uniformizowane CSV (koszt, czas, statystyki grup, gęstość, itp.).
  \item \textbf{Analiza} — skrypty w \texttt{scripts/analysis} generują wykresy porównawcze, fronty Pareto, heatmapy, profile wydajności oraz figury eksportowane do pracy.
\end{itemize}

Wariant \emph{dynamiczny} dodaje symulator mutacji grafu i krokową reoptymalizację, zachowując tę samą ścieżkę walidacji i zapisu. Dzięki temu eksperymenty statyczne i dynamiczne mają jednolity format danych i można je analizować tym samym narzędziem.
