\chapter{Wprowadzenie}

\section{Wstęp i motywacja}
W ostatnich latach coraz większe znaczenie zyskują modele subskrypcyjne w sektorze oprogramowania i usług cyfrowych. Zgodnie z indeksem gospodarki subskrypcyjnej rynek ten zwiększył swoją wartość o ponad 400\% od roku 2012 do roku 2021 \cite{subscriptionEconomyIndex}, a w 2024 roku osiągnął przybliżoną wartość blisko 600 miliardów dolarów \cite{subscriptionEconomyPrice2024}.

Konsumenci coraz częściej opłacają regularne abonamenty zamiast jednorazowych zakupów, co zapewnia firmom stałe przychody, a użytkownikom wygodny dostęp do usług. Wiele popularnych platform, w tym aplikacje edukacyjne, serwisy streamingowe czy oprogramowanie w modelu Software as a Service (SaaS), opiera się na modelu subskrypcyjnym. Serwisy streamingowe, takie jak Spotify czy Netflix, często oferują plany rodzinne, w których kilka osób może współdzielić jedną subskrypcję grupową, której koszt jest niższy niż zakup kilku licencji indywidualnych. W podobny sposób platforma edukacyjna do nauki języków Duolingo udostępnia plan rodzinny Super Duolingo (ang. Duolingo Super Family) \cite{duolingo_family}, który pozwala grupie znajomych lub osób spokrewnionych współdzielić korzyści subskrypcji premium. Zachęca to użytkowników do grupowego zakupu licencji, redukując koszt przypadający na jedną osobę.

W przypadku formowania się takich grup kontekst społeczny odgrywa istotną rolę. Rozsądne i optymalne korzystanie z subskrypcji grupowych wymaga, aby główny posiadacz subskrypcji znał osoby zainteresowane wspólnym korzystaniem z usługi --- ze względu na dzielenie kosztów lub wspólne zainteresowania. Rozwój mediów społecznościowych i komunikatorów ułatwia zawieranie takich porozumień w gronie znajomych lub osób o podobnych zainteresowaniach. W praktyce często dochodzi do sytuacji, w których użytkownicy umawiają się na wspólny zakup abonamentu. Sieć powiązań społecznych determinuje, kto z kim może skutecznie współdzielić licencję.

Analiza przedstawionych mechanizmów prowadzi do sformułowania problemu optymalizacyjnego: jak zaplanować zakup licencji w grupie powiązanych użytkowników, aby zminimalizować łączny koszt dostępu do usługi. Innymi słowy, mając daną sieć znajomości oraz dostępne opcje licencyjne, należy dobrać podzbiór użytkowników kupujących licencje (oraz rodzaj tych licencji) tak, by wszyscy użytkownicy mieli dostęp do usługi przy możliwie najniższym koszcie. Intuicyjnie jest to problem pokrycia grafu zbiorem \emph{właścicieli} (osób wykupujących licencje), w taki sposób, by każdy wierzchołek był albo objęty licencją, albo sąsiadował z kimś, kto licencję posiada.

Warto zauważyć, że opisana struktura problemu ma ścisłe powiązania z zagadnieniami teorii grafów. Sformułowanie jest bliskie klasycznemu problemowi dominowania w grafach, a w dalszej części pokażemy też związki z jego wariantem określanym jako \emph{dominowanie rzymskie}. Zależność ta stanowi ważny element prowadzonej analizy i uzupełnia główny cel pracy, którym jest optymalizacja kosztów licencji w społeczności użytkowników.

\section{Cele i zakres pracy}
Celem niniejszej pracy jest formalizacja i analiza problemu optymalnego zakupu licencji w sieciach społecznościowych, zaproponowanie metod jego rozwiązania oraz weryfikacja przyjętych rozwiązań. W pierwszej kolejności opracowany zostanie model grafowy opisujący powiązania między użytkownikami oraz różne strategie zakupowe wraz z odpowiadającymi im kosztami. Taki model pozwoli zdefiniować problem minimalizacji kosztów --- jako zadanie optymalizacyjne na grafie. Następnie wykażemy ścisły związek z problemem dominowania w grafach. W szczególności pokażemy, że dla pewnej klasy modeli licencjonowania zadanie optymalnego doboru subskrypcji jest równoważne znalezieniu minimalnego zbioru dominującego lub rozwiązaniu pokrewnego problemu \emph{dominowania rzymskiego}. Odwołanie do znanych wyników o dominowaniu obejmuje zarówno aspekty złożoności obliczeniowej, jak i badania nad algorytmami aproksymacyjnymi, co pozwala lepiej zrozumieć trudności badanego problemu oraz zaprojektować efektywne metody jego rozwiązywania.

Zakres pracy obejmuje analizę teoretyczną badanego problemu oraz metody algorytmiczne jego rozwiązania, uzupełnione o eksperymenty obliczeniowe. Rozpatrzone zostaną różne modele cenowe licencji, odzwierciedlające różnice między licencjami indywidualnymi a grupowymi. Uwzględnimy także scenariusze hipotetyczne odpowiadające wariantom nawiązującym do dominowania rzymskiego (np. przypadki, w których koszt licencji grupowej stanowi wielokrotność ceny licencji indywidualnej).

Osobno przeanalizujemy warianty, w których decyzje zakupowe podejmowane są globalnie (jednocześnie dla całej społeczności), oraz scenariusze dynamiczne. W wersji dynamicznej zakupy realizowane są w kolejnych krokach czasowych, a struktura sieci społecznościowej może się zmieniać (rozszerzanie lub zmniejszanie liczby użytkowników, powstawanie lub zanik relacji). Ze względu na wysoką złożoność obliczeniową problemu przedstawimy zarówno metody dokładne, jak i heurystyczne, które dostarczają rozwiązania dobrej jakości w czasie akceptowalnym praktycznie. Celem praktycznym jest wskazanie podejść skutecznych w optymalizacji kosztów subskrypcji w dużych sieciach społecznościowych oraz identyfikacja czynników najsilniej wpływających na wyniki, co zilustrujemy wynikami eksperymentów.

\section{Wkład pracy i struktura dokumentu}
\subsection*{Wkład pracy}
\begin{itemize}
  \item Sformalizowanie modelu optymalizacji zakupu licencji w sieciach społecznościowych: etykietowanie ról $f:V\to\{0,1,2\}$, warunki wykonalności (pokrycie, sąsiedztwo, pojemność) oraz funkcja kosztu.
  \item Wykazanie powiązania z \emph{dominowaniem rzymskim} i sformułowanie twierdzenia o równoważności w szczególnym przypadku kosztów/pojemności; omówienie konsekwencji złożonościowych i aproksymacyjnych.
  \item Implementacja i porównanie metod: ILP (PuLP/CBC), algorytm zachłanny, metaheurystyki (algorytm genetyczny, symulowane wyżarzanie, tabu search, ant colony optimization) oraz odmiany specyficzne (np. dla drzew).
  \item Opracowanie środowiska eksperymentalnego dla grafów syntetycznych i ego-sieci Facebooka, wraz z pomiarem czasu działania/kosztu, profilami wydajności i analizą granicy czasowej ILP.
  \item Analiza wariantu dynamicznego (mutacje grafu) i porównanie cold-start vs warm-start dla metaheurystyk.
  \item Studium wpływu polityk cenowych i typów planów (Individual/Duo/Family) na strukturę rozwiązań i koszt całkowity.
\end{itemize}

\subsection*{Struktura dokumentu}
\begin{description}
    \item \textbf{Rozdział 1} --- Wprowadzenie. Przedstawia tło problemu, motywację podjęcia tematu oraz znaczenie optymalizacji kosztów w kontekście współdzielenia licencji w sieciach społecznościowych. Określone są cele i zakres pracy oraz jej ogólna struktura.

    \item \textbf{Rozdział 2} --- Model grafowy i analiza problemu. Definiuje reprezentację sieci społecznościowej w postaci grafu oraz formalny opis problemu optymalizacji kosztów. Uwzględnione są przyjęte założenia, definicje pojęć oraz warianty wynikające z odmiennych modeli cenowych i ograniczeń harmonogramowych. Rozdział stanowi podstawę do dalszych rozważań algorytmicznych.

    \item \textbf{Rozdział 3} --- Związek z dominowaniem w grafach. Omawia pojęcie zbiorów dominujących i dominowania rzymskiego, pokazując, w jaki sposób badany problem można interpretować w tych kategoriach. Przedstawione są również aspekty złożoności obliczeniowej, w tym dowód NP-trudności, oraz wynikające z tego konsekwencje dla możliwości projektowania algorytmów.

    \item \textbf{Rozdział 4} --- Dane testowe. Opisuje rodzaje grafów wykorzystanych w eksperymentach: syntetyczne, generowane przy pomocy popularnych modeli (m.in. Barabási–Albert, Watts–Strogatz, grafy dwudzielne i pełne), oraz grafy rzeczywiste pochodzące z repozytoriów badawczych. Uwzględniono także sposób przygotowania instancji testowych oraz narzędzia użyte do wizualizacji sieci.

    \item \textbf{Rozdział 5} --- Metody algorytmiczne optymalizacji kosztów licencji. Rozdział skupia się na części obliczeniowej. Przedstawiona jest formalizacja problemu w postaci programu całkowitoliczbowego oraz opis metod dokładnych, które mogą znaleźć optymalne rozwiązania dla mniejszych instancji. Omówione są także podejścia przybliżone i heurystyczne.

    \item \textbf{Rozdział 6} --- Eksperymenty i analiza wyników. Przedstawia proces oceny algorytmów na przygotowanych danych testowych. Określone są kryteria porównawcze (m.in. czas działania, złożoność obliczeniowa, uzyskane koszty), a następnie zaprezentowane wyniki eksperymentów dla różnych typów grafów i ich skal. Analizowany jest wpływ parametrów algorytmów na efektywność i jakość uzyskiwanych rozwiązań.

    \item \textbf{Rozdział 7} --- Analiza dynamicznej wersji problemu. Rozważany jest scenariusz, w którym zakupy licencji odbywają się sekwencyjnie, a struktura sieci może ulegać zmianie w czasie. Opisane są możliwe adaptacje algorytmów do takiej sytuacji oraz przeprowadzone eksperymenty badające ich skuteczność. Poruszona jest także kwestia stabilności i elastyczności strategii w środowisku dynamicznym.

    \item \textbf{Rozdział 8} --- Rozszerzenia modelu. Omawia dodatkowe aspekty, które mogą wpływać na decyzje optymalizacyjne, takie jak polityki cenowe oraz zróżnicowanie typów licencji. Analizowane są również potencjalne ograniczenia modelu i możliwości jego dalszego uogólnienia.

    \item \textbf{Rozdział 9} --- Podsumowanie i wnioski. Zawiera syntetyczne zestawienie wyników pracy. Wskazuje, w jakim stopniu zrealizowane zostały założone cele, oraz proponuje kierunki dalszych badań, w tym rozwój modeli, udoskonalenia algorytmów oraz badania nad skalowalnością i praktycznymi zastosowaniami. Rozdział ten zamyka całość pracy, odpowiadając na pytania badawcze i wskazując, jak uzyskane rezultaty mogą zostać wykorzystane w praktyce.
\end{description}
