\chapter{Wprowadzenie}\label{chap:introduction}
\section{Wstęp i motywacja}
W ostatnich latach coraz większe znaczenie zyskują modele subskrypcyjne w sektorze oprogramowania i usług cyfrowych. Zgodnie z indeksem gospodarki subskrypcyjnej rynek ten zwiększył swoją wartość o ponad 400\% od roku 2012 do roku 2021 \cite{subscriptionEconomyIndex}, a w 2024 roku osiągnął przybliżoną wartość blisko 600 miliardów dolarów \cite{subscriptionEconomyPrice2024}. 

Model subskrypcyjny zastępuje zakupy jednorazowe. Zapewnia firmom powtarzalne przychody, a użytkownikom ciągły dostęp do usług. Wiele popularnych platform, w tym aplikacje edukacyjne, serwisy streamingowe czy oprogramowanie w modelu Software as a Service (SaaS), opiera się na modelu subskrypcyjnym. Serwisy streamingowe, takie jak Spotify i Netflix, oferują plany rodzinne, w których kilka osób współdziela jedną subskrypcję grupową o niższym koszcie jednostkowym niż równoważna liczba subskrypcji indywidualnych. Analogicznie platforma Duolingo udostępnia plan rodzinny Super Duolingo (ang. Duolingo Super Family) \cite{duolingo_family}, który umożliwia grupie użytkowników współdzielenie korzyści subskrypcji premium. Mechanizm ten sprzyja zakupom grupowym i obniża koszt przypadający na użytkownika. W pracy terminy "subskrypcja" i "licencja" traktowane są równoważnie jako czasowe prawa dostępu.

Kontekst społeczny jest kluczowy przy formowaniu grup. Efektywne korzystanie z subskrypcji grupowych wymaga, aby główny posiadacz subskrypcji identyfikował użytkowników zainteresowanych współdzieleniem usługi, np. z powodu podziału kosztów, zbieżnych zainteresowań lub bliskich relacji. Grupy tworzone są głównie wśród krewnych i znajomych, co ułatwia ustanowienie relacji zaufania przy współdzieleniu konta. Dodatkowym czynnikiem jest wygoda: jedna wspólna subskrypcja upraszcza rozliczenia i zapewnia wszystkim członkom grupy jednolity dostęp do usługi. Czynniki społeczne wpływają na strukturę współdzielenia w sieciach użytkowników. Media społecznościowe i komunikatory ułatwiają zawieranie porozumień wśród użytkowników powiązanych. W praktyce użytkownicy koordynują wspólny zakup abonamentu. Sieć powiązań społecznych determinuje, kto z kim może skutecznie współdzielić licencję.

Problem optymalizacyjny polega na wyznaczeniu planu zakupu licencji w grupie powiązanych użytkowników, który minimalizuje łączny koszt dostępu do usługi. Równoważnie, dla danej sieci znajomości i katalogu opcji licencyjnych należy wybrać podzbiór nabywców licencji oraz typy licencji tak, aby wszyscy użytkownicy uzyskali dostęp do usługi przy minimalnym koszcie. Formalnie odpowiada to problemowi pokrycia wierzchołków przez zbiór nabywców licencji tak, aby każdy wierzchołek był objęty licencją lub sąsiadował z wierzchołkiem posiadającym licencję.

Struktura problemu ma ścisłe powiązania z zagadnieniami teorii grafów. Jest ona zbieżna z klasycznym problemem dominowania, ponieważ wybór nabywców licencji grupowych pełni funkcję zbioru dominującego. W obu przypadkach chodzi o to, aby wybrany zestaw wierzchołków pokrywał całą sieć. W rozdziale \ref{chap:roman_domination} wykazano związek z wariantem dominowania rzymskiego. Odniesienia te wspierają cel pracy, którym jest optymalizacja kosztów licencji w społeczności użytkowników.

\section{Przegląd istniejących rozwiązań}

Dotychczas w literaturze brak jest opracowań bezpośrednio analizujących problem optymalnego podziału licencji w sieciach społecznościowych. Zagadnienie minimalizacji kosztów zakupu planów grupowych w sieciach powiązanych użytkowników nie zostało dotąd opisane. Niniejsza praca formułuje i analizuje ten problem w ramach teorii grafów.

Najbliższą analogią jest klasyczne zagadnienie zbioru dominującego, które polega na znalezieniu minimalnego podzbioru wierzchołków, tak by każdy wierzchołek grafu należał do tego zbioru lub był jego sąsiadem. Problem ten jest NP-zupełny, co uzasadnia badania algorytmiczne i heurystyczne~\cite{garey1979}. Szczególne znaczenie w kontekście analizowanego problemu ma dominowanie rzymskie, które zostało szczegółowo opisane w rozdziale \ref{chap:roman_domination}. Koncepcja ta znalazła szerokie zastosowanie i liczne rozszerzenia, m.in. dominowanie włoskie i dominowanie zabezpieczone \cite{Roman2DominationSurvey}.

Model ten jest adekwatny do odwzorowania problemu podziału licencji. Standardowa wersja dominowania rzymskiego zakłada brak ograniczeń co do liczby sąsiadów, których może dominować wierzchołek z etykietą $2$. W praktyce plany subskrypcyjne narzucają limity (np. maksymalnie $6$ osób w planie rodzinnym). Prowadzi to do modelu dominowania z pojemnością, w którym każdy wierzchołek dominujący ma przypisaną wartość określającą liczbę sąsiadów objętych dominowaniem \cite{CapDom}.

Problem podziału licencji stanowi wariant zagadnienia dominowania w grafach, wykorzystujący idee dominowania rzymskiego oraz jego modyfikacje z ograniczeniami pojemności. Brak specjalistycznych opracowań wskazuje na lukę badawczą. Jednocześnie bogata literatura dotycząca dominowania w grafach dostarcza ugruntowanych narzędzi teoretycznych i algorytmicznych, co pozwala formalnie wykazać NP-trudność badanego problemu oraz zastosować zarówno metody dokładne (np. programowanie całkowitoliczbowe), jak i przybliżone heurystyki, wzorując się na podejściach znanych z teorii dominowania \cite{Roman2DominationSurvey, CapDom}.


\section{Cele i zakres pracy}
Celem niniejszej pracy jest formalizacja i analiza problemu optymalnego zakupu licencji w sieciach społecznościowych, zaproponowanie metod jego rozwiązania oraz weryfikacja przyjętych rozwiązań. W pierwszej kolejności opracowany został model grafowy opisujący powiązania między użytkownikami oraz różne strategie zakupowe wraz z odpowiadającymi im kosztami. Model ten umożliwia zdefiniowanie problemu minimalizacji kosztów jako zadania optymalizacyjnego na grafie. Następnie wykazano ścisły związek z problemem dominowania w grafach. W szczególności pokazano, że dla pewnej klasy modeli licencjonowania zadanie optymalnego doboru subskrypcji jest równoważne znalezieniu minimalnego zbioru dominującego lub rozwiązaniu pokrewnego problemu dominowania rzymskiego. Odwołanie do znanych wyników o dominowaniu obejmuje zarówno aspekty złożoności obliczeniowej, jak i badania nad algorytmami aproksymacyjnymi, co pozwala lepiej zrozumieć trudności badanego problemu oraz zaprojektować efektywne metody jego rozwiązywania.

Zakres pracy obejmuje analizę teoretyczną badanego problemu oraz metody algorytmiczne jego rozwiązania, uzupełnione o eksperymenty obliczeniowe. Rozpatrzone zostały różne modele cenowe licencji, zarówno hipotetyczne, jak i rzeczywiste. Do pierwszej grupy należą warianty nawiązujące do dominowania rzymskiego, w których koszt licencji grupowej stanowi wielokrotność ceny licencji indywidualnej. W drugiej grupie znajdują się modele oparte na rzeczywistych ofertach usług, takich jak Spotify, Netflix i Duolingo, co pozwala weryfikować wyniki w kontekście praktycznych scenariuszy.


Osobno przeanalizowane zostały warianty, w których decyzje zakupowe podejmowane są globalnie (jednocześnie dla całej społeczności), oraz scenariusze dynamiczne. W wersji dynamicznej zakupy realizowane są w kolejnych krokach czasowych, a struktura sieci społecznościowej może się zmieniać (rozszerzanie lub zmniejszanie liczby użytkowników, powstawanie lub zanik relacji). Ze względu na wysoką złożoność obliczeniową problemu przedstawione zostały zarówno metody dokładne, jak i heurystyczne. Metody dokładne gwarantują znalezienie rozwiązania optymalnego, lecz ich czas działania szybko rośnie wraz z rozmiarem grafu. Metody heurystyczne nie gwarantują optymalności, lecz dostarczają rozwiązania dobrej jakości w czasie akceptowalnym obliczeniowo. Celem praktycznym jest wskazanie podejść skutecznych w optymalizacji kosztów subskrypcji w dużych sieciach społecznościowych oraz identyfikacja czynników najsilniej wpływających na wyniki, co zilustrowane jest wynikami eksperymentów.

Podsumowując, w ramach pracy zrealizowane zostały następujące zadania badawcze i implementacyjne:
\begin{itemize}
  \item Sformalizowano model optymalizacji zakupu licencji w sieciach społecznościowych, obejmujący etykietowanie ról $f:V\to\{0,1,2\}$, warunki wykonalności (pokrycie, sąsiedztwo, pojemność) oraz funkcję kosztu.
  \item Wykazano powiązania z dominowaniem rzymskim i sformułowano twierdzenie o równoważności w szczególnym przypadku kosztów i pojemności; omówiono również konsekwencje złożonościowe i aproksymacyjne.
  \item Zaimplementowano i porównano różne metody: ILP (PuLP/CBC), algorytm zachłanny, metaheurystyki (algorytm genetyczny, symulowane wyżarzanie, przeszukiwanie tabu, algorytm mrówkowy).
  \item Opracowano środowisko eksperymentalne dla grafów syntetycznych i ego-sieci platformy Facebook, wraz z pomiarem czasu działania i kosztu.
  \item Przeanalizowano wariant dynamiczny (mutacje grafu) oraz porównano podejścia cold-start i warm-start dla metaheurystyk.
  \item Zbadano wpływ polityk cenowych i typów planów (Individual/Duo/Family) na strukturę rozwiązań i koszt całkowity.
\end{itemize}

\section{Struktura}

Struktura pracy obejmuje dziewięć rozdziałów. Obejmują one część wprowadzającą, w której przedstawiono tło i motywację podjętego zagadnienia, część analityczno-badawczą, zawierającą opis zaproponowanych modeli oraz przeprowadzonych eksperymentów, a także część podsumowującą, w której sformułowano wnioski oraz wskazano możliwe kierunki dalszych badań. Poniżej zaprezentowano opis treści wszystkich rozdziałów, stanowiących kolejne etapy realizacji pracy.

\begin{description}
  \item \textbf{Rozdział 1} -- Wprowadzenie. Przedstawia tło problemu, motywację podjęcia tematu oraz znaczenie optymalizacji kosztów w kontekście współdzielenia licencji w sieciach społecznościowych. Określone są cele i zakres pracy oraz jej ogólna struktura.

  \item \textbf{Rozdział 2} -- Model grafowy i analiza problemu. Definiuje reprezentację sieci społecznościowej w postaci grafu oraz formalny opis problemu optymalizacji kosztów. Uwzględnione są przyjęte założenia, definicje pojęć oraz warianty wynikające z odmiennych modeli cenowych i ograniczeń pojemnościowych. Rozdział stanowi podstawę do dalszych rozważań algorytmicznych.

  \item \textbf{Rozdział 3} -- Związek z dominowaniem w grafach. Omawia pojęcie zbiorów dominujących i dominowania rzymskiego, pokazując, w jaki sposób badany problem można interpretować w tych kategoriach. Przedstawione są również aspekty złożoności obliczeniowej, w tym dowód NP-trudności, oraz wynikające z tego konsekwencje dla możliwości projektowania algorytmów.

  \item \textbf{Rozdział 4} -- Dane testowe. Opisuje rodzaje grafów wykorzystanych w eksperymentach: syntetyczne, generowane przy pomocy popularnych modeli (Barabási--Albert, Watts--Strogatz oraz Erdős--Rényi), oraz grafy rzeczywiste pochodzące z repozytoriów badawczych.

  \item \textbf{Rozdział 5} -- Metody algorytmiczne optymalizacji kosztów licencji. Rozdział skupia się na części obliczeniowej. Przedstawiona jest formalizacja problemu w postaci programu całkowitoliczbowego oraz opis metod dokładnych, które mogą znaleźć optymalne rozwiązania dla mniejszych instancji. Omówione są także podejścia heurystyczne oraz metaheurystyczne, które umożliwiają znalezienie dobrych rozwiązań w rozsądnym czasie dla większych grafów. Każda metoda jest szczegółowo opisana pod kątem implementacji i parametrów.

  \item \textbf{Rozdział 6} -- Eksperymenty i analiza wyników. Przedstawia proces oceny algorytmów na przygotowanych danych testowych. Określone są kryteria porównawcze (m.in. czas działania, złożoność obliczeniowa, uzyskane koszty), a następnie zaprezentowane wyniki eksperymentów dla różnych typów grafów i ich skal. Analizowany jest wpływ parametrów algorytmów na efektywność i jakość uzyskiwanych rozwiązań.

  \item \textbf{Rozdział 7} -- Analiza dynamicznej wersji problemu. Rozważany jest scenariusz, w którym zakupy licencji odbywają się sekwencyjnie, a struktura sieci może ulegać zmianie w czasie. Opisane są możliwe adaptacje algorytmów do takiej sytuacji oraz przeprowadzone eksperymenty badające ich skuteczność. Poruszona jest także kwestia stabilności i elastyczności strategii w środowisku dynamicznym.

  \item \textbf{Rozdział 8} -- Rozszerzenia modelu. Omawia dodatkowe aspekty, które mogą wpływać na decyzje optymalizacyjne, takie jak polityki cenowe oraz zróżnicowanie typów licencji. Analizowane są również potencjalne ograniczenia modelu i możliwości jego dalszego uogólnienia.

  \item \textbf{Rozdział 9} -- Podsumowanie i wnioski. Zawiera syntetyczne zestawienie wyników pracy. Wskazuje, w jakim stopniu zrealizowane zostały założone cele, oraz proponuje kierunki dalszych badań, w tym rozwój modeli, udoskonalenia algorytmów oraz badania nad skalowalnością i praktycznymi zastosowaniami. Rozdział ten zamyka całość pracy, odpowiadając na pytania badawcze i wskazując, jak uzyskane rezultaty mogą zostać wykorzystane w praktyce.
\end{description}
