\chapter{Wprowadzenie}

\section{Wstęp i motywacja}
W ostatnich latach coraz większe znaczenie zyskują modele subskrypcyjne w sektorze oprogramowania i usług cyfrowych. Zgodnie z indeksem gospodarki subskrypcyjnej rynek ten zwiększył swoją wartość o ponad 400\% od roku 2012 do roku 2021 \cite{subscriptionEconomyIndex} oraz osiągając w 2024 przybliżoną wartość prawie 600 miliardów dolarów \cite{subscriptionEconomyPrice2024}. Konsumenci coraz częściej opłacają regularne abonamenty zamiast jednorazowych zakupów, co zapewnia firmom stałe przychody, a użytkownikom wygodny dostęp do usług. Wiele popularnych platform, w tym aplikacje edukacyjne, serwisy streamingowe czy oprogramowanie SaaS, opiera się na modelu subskrypcyjnym. Serwisy streamingowe takie jak Spotify często oferują plany rodzinne, w których kilka osób może współdzielić jedną subskrypcję grupową, której koszt jest sumarycznie niższy od zakupu kilku licencji indywidualnych. W podobny sposób platforma edukacyjna do nauki języków Duolingo udostępnia plan Rodzina Super Duolingo (ang. Duolingo Super Family) \cite{duolingo_family}, która pozwala grupie znajomych lub osób spokrewnionych na współdzielenie korzyści subskrypcji premium. Zachęca to użytkowników do grupowego zakupu licencji, redukując koszt przypadający na jedną osobę.

W przypadku formowania się tych właśnie grup kontekst społeczny odgrywa bardzo istotną rolę. Rozsądne i optymalne korzystanie z subskrypcji grupowych wymaga, aby główny posiadacz subskrypcji znał osoby zainteresowane wspólnym korzystaniem z usługi czy to ze względu na chęć dzielenia kosztów, czy wspólną pasję. Rozwój mediów społecznościowych i komunikatorów ułatwia zawieranie takich porozumień w gronie znajomych lub osób o podobnych zainteresowaniach. W praktyce często dochodzi do sytuacji, w których użytkownicy umawiają się na wspólny zakup abonamentu. Sieć powiązań społecznych decyduje o tym, kto z kim może efektywnie współdzielić licencję.

Analiza przedstawionych mechanizmów prowadzi do sformułowania problemu optymalizacyjnego polegającego na takim zaplanowaniu zakupu licencji w grupie powiązanych użytkowników, aby zminimalizować łączny koszt dostępu do usługi przez użytkowników. Innymi słowy, mając daną sieć znajomości oraz dostępne opcje licencyjne, jak dobrać podzbiór użytkowników kupujących licencje (oraz rodzaj tych licencji), by wszyscy użytkownicy mieli dostęp do usługi przy możliwie najniższym sumarycznym koszcie. Intuicyjnie, jest to problem pokrycia grafu pewnym zbiorem "liderów" (osób wykupujących licencje) w taki sposób, by każdy w grafie był albo sam licencjonowany, albo połączony z kimś, kto licencję posiada.

Warto zauważyć, że opisana struktura problemu znajduje również odzwierciedlenie w zagadnieniach teorii grafów. Jej sformułowanie jest bliskie klasycznemu problemowi dominowania w grafach, a przy dalszej analizie uwypukli się także, że można powiązać to z jego wariantem określanym jako dominowanie rzymskie. Zależność ta stanowi istotny element prowadzonej analizy, a jednocześnie uzupełnia istotny cel pracy, którym jest optymalizacja kosztów licencji w społeczności użytkowników.

\section{Cele i zakres pracy}
Celem niniejszej pracy jest formalizacja i analiza problemu optymalnego zakupu licencji w sieciach społecznościowych, zaproponowanie metod jego rozwiązania oraz sprawdzenie przyjętych rozwiązań. W pierwszej kolejności opracowany zostanie model grafowy opisujący powiązania między użytkownikami oraz różne strategie zakupowe wraz z odpowiadającymi im kosztami. Taki model pozwoli zdefiniować problem minimalizacji kosztów - jako zadanie optymalizacyjne na grafie. Następnym celem pracy jest wykazanie, że problem ten jest ściśle powiązany z problemem dominowania w grafach, znanym z teorii grafów. W szczególności pokazane zostanie, że dla pewnej klasy modeli licencjonowania zadanie optymalnego doboru subskrypcji jest równoważne znalezieniu tak zwanego zbioru dominującego minimalnej wielkości lub rozwiązaniu pokrewnego problemu dominacji rzymskiej. Wyznaczenie ich wzajemnej korelacji umożliwi odwołanie się do znanych wyników dotyczących problemów dominowania w grafach, obejmujących zarówno aspekty złożoności obliczeniowej, jak i badania nad algorytmami aproksymacyjnymi, co pozwoli lepiej zrozumieć trudności badanego problemu oraz zaprojektować efektywne metody jego rozwiązywania.

Zakres pracy obejmuje zarówno analizę teoretyczną badanego problemu, jak i rozważania nad metodami algorytmicznymi jego rozwiązania, uzupełnione o eksperymenty obliczeniowe. Rozpatrzone zostaną różne modele cenowe licencji, odzwierciedlające rzeczywiste różnice między licencjami indywidualnymi a grupowymi. 
Uwzględnione będą także wybrane scenariusze hipotetyczne dla wariantu imitującego dominowanie rzymskie, między innymi takie, w których koszt licencji grupowej stanowi wielokrotność ceny licencji indywidualnej (np. dwukrotność lub trzykrotność). 
Analizie poddane zostaną warianty problemu, w których decyzje dotyczące zakupu licencji podejmowane są globalnie, czyli jednocześnie dla całej społeczności użytkowników, a także scenariusze dynamiczne. W scenariuszu dynamicznym zakupy mogą być realizowane w kolejnych krokach czasowych, a struktura sieci społecznościowej może ulegać zmianom poprzez rozszerzanie lub zmniejszanie się liczby użytkowników oraz powstawanie lub zanikanie relacji między nimi. Ze względu na wysoką złożoność obliczeniową problemu, w pracy przedstawiony zostanie przegląd potencjalnych algorytmów zarówno dokładnych, jak i heurystycznych, które mogą dostarczać rozwiązania dobrej jakości w czasie obliczeniowym akceptowalnym z praktycznego punktu widzenia. Celem praktycznym jest wskazanie podejść skutecznych w optymalizacji kosztów subskrypcji w dużych sieciach społecznościowych oraz identyfikacja czynników mających największy wpływ na końcowe wyniki optymalizacji, co zostanie zilustrowane wynikami eksperymentów.

\section{Struktura pracy}

Struktura pracy została zorganizowana w dziewięciu rozdziałach. Obejmują one część wprowadzającą, w której przedstawiono tło i motywację podjętego zagadnienia, część analityczno–badawczą, zawierającą opis zaproponowanych modeli oraz przeprowadzonych eksperymentów, a także część podsumowującą, w której sformułowano wnioski oraz wskazano możliwe kierunki dalszych badań. Poniżej zaprezentowano opis treści wszystkich rozdziałów, stanowiących kolejne etapy realizacji pracy.
\\
% \begin{description}
%     \item \textbf{Rodział 1} --- Przedstawia tło problemu, motywację podjęcia tematu, cele oraz zakres pracy, a także strukturę całej pracy.
%     \item \textbf{Rodział 2} --- Zawiera formalny opis modelu grafowego dla sieci społecznościowej oraz definicję analizowanego problemu optymalizacji kosztów. Omówiono tam przyjęte założenia oraz różne warianty problemu wynikające z modeli cenowych i harmonogramu zakupów.
%     \item \textbf{Rodział 3} --- Dotyczy pojęcia dominowania w grafach i jego związku z naszym problemem. Wprowadzone zostają definicje zbioru dominującego oraz dominowania rzymskiego, a następnie pokazana jest interpretacja naszego problemu w tych kategoriach. Rozdział ten porusza też kwestie złożoności obliczeniowej - wykazuje NP-trudność problemu oraz omawia jej konsekwencje dla dalszych analiz.
%     \item \textbf{Rodział 4} --- Skupia się na algorytmicznych aspektach problemu. Przedstawiona zostaje formalizacja zadania w postaci programu całkowitoliczbowego oraz omówione są znane w literaturze metody dokładne znajdowania minimalnych zbiorów dominujących. Rozdział ten zawiera również przegląd wybranych algorytmów przybliżonych i heurystyk, które potencjalnie można zastosować do dużych grafów społecznościowych.
%     \item \textbf{Rodział 5} ---  Prezentuje proponowane rozwiązania heurystyczne opracowane w ramach pracy. Opisane zostaną autorskie algorytmy heurystyczne dostosowane do specyfiki problemu licencji. Zaprezentowane zostaną również ewentualne algorytmy wyszukiwania lokalnego i strategie poprawy znalezionych rozwiązań.
%     \item \textbf{Rodział 6} ---  Rozszerza analizę na wersje dynamiczne problemu. Rozważane jest scenariusz, w którym zakupy licencji dokonywane są w sekwencji, a sieć użytkowników może się zmieniać. Omówiony zostaje wpływ takiej dynamiki na strategię optymalną oraz ewentualne podejścia algorytmiczne.
%     \item \textbf{Rodział 7} --- Zawiera opis przeprowadzonych eksperymentów i symulacji. Przedstawione zostaną wyniki testów algorytmów na przykładowych grafach symulujących sieci społeczne. Analizie poddano różne konfiguracje, a uzyskane wyniki zestawiono pod kątem jakości rozwiązań i czasów obliczeń.
%     \item \textbf{Rodział 8} --- Interpretowane są obserwacje poczynione na podstawie eksperymentów - na przykład jak struktura sieci wpływa na oszczędności kosztów, jaka jest efektywność poszczególnych algorytmów, w jakich warunkach współdzielenie licencji przynosi największe korzyści. Poruszone zostają także aspekty praktyczne wdrożenia strategii grupowego zakupu subskrypcji oraz potencjalne ograniczenia.
%     \item \textbf{Rodział 9} --- Zawiera wnioski wynikające z przeprowadzonych badań, podkreśla osiągnięte cele oraz proponuje kierunki dalszych badań. Wskazane są możliwe usprawnienia modeli oraz rozwinięcia algorytmów. Rozdział ten kończy pracę, syntetycznie odpowiadając na pytanie sformułowane na początku i sugerując, jak uzyskane rezultaty mogą zostać wykorzystane w praktyce.
% \end{description}

\begin{description}
    \item \textbf{Rozdział 1} --- Wprowadzenie. Przedstawia tło problemu, motywację podjęcia tematu oraz znaczenie optymalizacji kosztów w kontekście współdzielenia licencji w sieciach społecznościowych. Określone są cele i zakres pracy oraz jej ogólna struktura.
    
    \item \textbf{Rozdział 2} --- Model grafowy i analiza problemu. Definiuje reprezentację sieci społecznościowej w postaci grafu oraz formalny opis problemu optymalizacji kosztów. Uwzględnione są przyjęte założenia, definicje pojęć oraz warianty wynikające z odmiennych modeli cenowych i ograniczeń harmonogramowych. Rozdział stanowi podstawę do dalszych rozważań algorytmicznych.
    
    \item \textbf{Rozdział 3} --- Związek z dominowaniem w grafach. Omawia pojęcie zbiorów dominujących i dominowania rzymskiego, pokazując, w jaki sposób badany problem można interpretować w tych kategoriach. Przedstawione są również aspekty złożoności obliczeniowej, w tym dowód NP-trudności, oraz wynikające z tego konsekwencje dla możliwości projektowania algorytmów.
    
    \item \textbf{Rozdział 4} --- Dane testowe. Opisuje rodzaje grafów wykorzystanych w eksperymentach: syntetyczne, generowane przy pomocy popularnych modeli (m.in. Barabási–Albert, Watts–Strogatz, grafy dwudzielne i pełne), oraz grafy rzeczywiste pochodzące z repozytoriów badawczych. Uwzględniono także sposób przygotowania instancji testowych oraz narzędzia użyte do wizualizacji sieci.
    
    \item \textbf{Rozdział 5} --- Metody algorytmiczne optymalizacji kosztów licencji. Rozdział skupia się na części obliczeniowej. Przedstawiona jest formalizacja problemu w postaci programu całkowitoliczbowego oraz opis metod dokładnych, które mogą znaleźć optymalne rozwiązania dla mniejszych instancji. Omówione są także podejścia przybliżone i heurystyczne. 
    
    \item \textbf{Rozdział 6} --- Eksperymenty i analiza wyników. Przedstawia proces oceny algorytmów na przygotowanych danych testowych. Określone są kryteria porównawcze (m.in. czas działania, złożoność obliczeniowa, uzyskane koszty), a następnie zaprezentowane wyniki eksperymentów dla różnych typów grafów i ich skal. Analizowany jest wpływ parametrów algorytmów na efektywność i jakość uzyskiwanych rozwiązań.
    
    \item \textbf{Rozdział 7} --- Analiza dynamicznej wersji problemu. Rozważany jest scenariusz, w którym zakupy licencji odbywają się sekwencyjnie, a struktura sieci może ulegać zmianie w czasie. Opisane są możliwe adaptacje algorytmów do takiej sytuacji oraz przeprowadzone eksperymenty badające ich skuteczność. Poruszona jest także kwestia stabilności i elastyczności strategii w środowisku dynamicznym.
    
    \item \textbf{Rozdział 8} --- Rozszerzenia modelu. Omawia dodatkowe aspekty, które mogą wpływać na decyzje optymalizacyjne, takie jak polityki cenowe czy zróżnicowanie typów licencji. Analizowane są również potencjalne ograniczenia modelu i możliwości jego dalszego uogólnienia.
    
    \item \textbf{Rozdział 9} --- Podsumowanie i wnioski. Zawiera syntetyczne zestawienie wyników pracy. Wskazuje, w jakim stopniu zrealizowane zostały założone cele, oraz proponuje kierunki dalszych badań, w tym rozwój modeli, udoskonalenia algorytmów oraz badania nad skalowalnością i praktycznymi zastosowaniami. Rozdział ten zamyka całość pracy, odpowiadając na pytania badawcze i wskazując, jak uzyskane rezultaty mogą zostać wykorzystane w praktyce.
\end{description}
