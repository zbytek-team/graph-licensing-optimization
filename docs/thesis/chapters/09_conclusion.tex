\chapter{Podsumowanie i wnioski końcowe}

\section{Podsumowanie pracy}

Celem niniejszej pracy magisterskiej było opracowanie oraz analiza grafowego modelu optymalizacji kosztów zakupu licencji oprogramowania (na przykładzie usług subskrypcyjnych takich jak \textit{Duolingo Super} czy \textit{Spotify Premium}) w sieciach społecznościowych. Problem ten przedstawiono jako rozszerzenie klasycznych problemów dominowania w grafach, w szczególności dominowania rzymskiego, uwzględniając dodatkowe ograniczenia dotyczące maksymalnej wielkości grup oraz specyficzne warunki kosztowe.

W ramach realizacji celu pracy:

\begin{itemize}
    \item Zdefiniowano formalny model grafowy reprezentujący użytkowników oraz relacje społeczne, a także dokładne założenia dotyczące zakupu licencji indywidualnych oraz grupowych.
    \item Dokonano szczegółowej analizy powiązań między analizowanym problemem a klasycznym oraz rzymskim dominowaniem, wskazując jednoznacznie na jego NP-trudność.
    \item Opracowano oraz zaimplementowano szeroki wachlarz algorytmów rozwiązujących problem: od metod dokładnych (MIP) przez heurystyki zachłanne po metaheurystyki (algorytmy genetyczne, tabu search, symulowane wyżarzanie, ant colony optimization).
    \item Przeprowadzono eksperymenty wykorzystujące zarówno dane rzeczywiste (\textit{SNAP, NetworkRepository}), jak i syntetyczne (\textit{Barabási–Albert, Erdős–Rényi, Watts–Strogatz}).
    \item Przeprowadzono analizę dynamicznej wersji problemu, uwzględniając zmieniającą się strukturę sieci społecznościowej.
    \item Przedstawiono rozszerzenia modelu obejmujące analizę różnych scenariuszy cenowych oraz wprowadzenie dodatkowych typów subskrypcji (np. Duo, Student).
\end{itemize}

\section{Najważniejsze wnioski}

Najważniejsze wnioski płynące z przeprowadzonych badań to:

\begin{itemize}
    \item Problem optymalizacji zakupu licencji w grafach społecznościowych stanowi uogólnienie klasycznego problemu dominowania rzymskiego i jest silnie NP-trudny. Metody dokładne mają ograniczone zastosowanie do niewielkich instancji problemu.
    \item Algorytmy zachłanne zapewniają szybkie i akceptowalne jakościowo rozwiązania dla dużych grafów, choć ustępują metaheurystykom pod względem jakości.
    \item Metaheurystyki (w szczególności algorytmy genetyczne, tabu search, symulowane wyżarzanie i ACO) umożliwiają uzyskanie rozwiązań bliskich optymalnym przy rozsądnym czasie działania.
    \item Dynamiczna wersja problemu jest istotna w praktycznych zastosowaniach, gdzie graf społecznościowy ulega zmianom; podejścia z warm-startem zapewniają dobrą stabilność i krótszy czas reakcji.
    \item Wprowadzenie dodatkowych typów licencji (Duo, Student) oraz zmiany polityk cenowych istotnie wpływają na optymalne rozwiązania i pozwalają na obniżenie kosztów użytkowników oraz zwiększenie przychodów dostawców usług.
\end{itemize}

\section{Wnioski praktyczne i rekomendacje}

Z perspektywy praktycznej analizowany model może być użyteczny zarówno dla użytkowników platform subskrypcyjnych, jak i dostawców tych usług:

\begin{itemize}
    \item \textbf{Dla użytkowników:} Możliwość znaczącego zmniejszenia kosztów korzystania z usług poprzez efektywne organizowanie grup zakupowych zgodnie z analizowanymi strategiami optymalizacyjnymi.
    \item \textbf{Dla dostawców usług:} Wykorzystanie wniosków z tej pracy do ustalania efektywnych polityk cenowych oraz wprowadzania nowych, dostosowanych do potrzeb użytkowników, planów subskrypcyjnych.
\end{itemize}

\section{Ograniczenia i wyzwania przyszłych badań}

Wśród głównych ograniczeń niniejszej pracy należy wymienić:

\begin{itemize}
    \item Ograniczone możliwości stosowania metod dokładnych (MIP) do grafów o dużych rozmiarach.
    \item Trudność analizy globalnej optymalności uzyskiwanych rozwiązań ze względu na NP-trudność problemu.
    \item Uproszczenia przyjęte w modelu społecznym (brak uwzględnienia bardziej złożonych dynamik społecznych i preferencji użytkowników).
\end{itemize}

Wyzwania przyszłych badań obejmują:

\begin{itemize}
    \item Rozwój bardziej efektywnych metod metaheurystycznych oraz hybrydowych, umożliwiających uzyskiwanie jeszcze lepszych rozwiązań w krótszym czasie.
    \item Uwzględnienie aspektów behawioralnych użytkowników (np. preferencje, chęć współdzielenia licencji).
    \item Dalszą analizę dynamiczną z uwzględnieniem bardziej realistycznych scenariuszy zmian społecznych.
\end{itemize}

\section{Możliwości dalszego rozwoju modelu}

Potencjalne kierunki rozwoju modelu obejmują między innymi:

\begin{itemize}
    \item Integrację metod uczenia maszynowego i grafowych sieci neuronowych (GNN) do efektywnego rozwiązywania problemów optymalizacji w dużych sieciach społecznych.
    \item Rozszerzenie analizowanych scenariuszy o dodatkowe typy licencji i pakietów usług z różnych branż (np. subskrypcje treści multimedialnych, usługi chmurowe).
    \item Badanie wpływu zachowań użytkowników i mechanizmów zachęcających ich do współdzielenia licencji (np. zachęty finansowe, programy lojalnościowe).
\end{itemize}

\section{Podsumowanie rozdziału}

W niniejszym rozdziale dokonano kompleksowego podsumowania osiągniętych rezultatów. Wskazano główne osiągnięcia pracy oraz praktyczne implikacje wyników. Omówiono także ograniczenia przyjętych założeń oraz sformułowano możliwe kierunki dalszych badań, podkreślając potencjał praktyczny i naukowy opracowanego modelu optymalizacji kosztów zakupu licencji w sieciach społecznościowych.
