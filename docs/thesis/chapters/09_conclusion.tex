\chapter{Podsumowanie i wnioski końcowe}

\section{Podsumowanie pracy}
Przedstawiono model optymalizacji kosztów licencji w sieciach społecznych oparty na reprezentacji grafowej. Zdefiniowano rodzinę typów licencji z ograniczeniami pojemności oraz sformalizowano funkcję celu minimalizującą łączny koszt. Wykazano powiązanie z problemem dominowania oraz dominowania rzymskiego w grafach, co motywuje zarówno trudność obliczeniową (NP-trudność), jak i dobór metod algorytmicznych.

Zaimplementowano i porównano zestaw metod: solver ILP dla małych instancji (punkt odniesienia), heurystyki konstrukcyjne (zachłanny, zbiór dominujący, losowy) oraz metaheurystyki (genetyczny, mrówkowy, tabu, symulowane wyżarzanie). Eksperymenty wykonano na grafach syntetycznych (losowe, małoświatowe, bezskalowe) i na sieciach rzeczywistych (ego-Facebook). Osobno przeanalizowano scenariusz dynamiczny, w którym graf ulega mutacjom w kolejnych krokach, oraz rozszerzenia modelu cenowego (warianty \texttt{roman\_p\_x}, konfiguracja \texttt{spotify}).

Powstał zestaw narzędzi analitycznych do automatycznej agregacji wyników, generowania wykresów i tabel oraz eksportu figur do części tekstowej. Wyniki zebrano i omówiono w rozdziałach eksperymentalnych.

\section{Najważniejsze wnioski}
\begin{itemize}
  \item \textbf{Złożoność i punkt odniesienia.} Problem jest co najmniej tak trudny jak dominowanie rzymskie, co potwierdza potrzebę metod przybliżonych dla średnich i dużych instancji. ILP stanowi wiarygodny punkt odniesienia dla małych grafów; dla większych szybko przekracza limity czasu.
  \item \textbf{Jakość vs czas.} Metaheurystyki osiągają najniższe koszty (zwłaszcza genetyczny i wyżarzanie), lecz są wolniejsze. Algorytm zachłanny jest dobrym kompromisem: bardzo szybki i tylko nieznacznie gorszy kosztowo w wielu przypadkach. Heurystyka zbioru dominującego plasuje się pomiędzy.
  \item \textbf{Wpływ struktury grafu.} Grafy bezskalowe sprzyjają niższym kosztom (huby umożliwiają większe grupy). Grafy losowe i małoświatowe są trudniejsze kosztowo. Trendy te są spójne między scenariuszami statycznym i dynamicznym.
  \item \textbf{Dynamika i warm start.} W scenariuszu dynamicznym umiarkowane mutacje prowadzą do stopniowego wzrostu trudności. Metaheurystyki z warm start utrzymują koszt stabilniej (a czasem poprawiają) względem przebiegów cold, co potwierdza korzyści z projekcji poprzedniego rozwiązania.
  \item \textbf{Konfiguracje licencji.} Dla \texttt{roman\_p\_x} koszt na wierzchołek rośnie monotonicznie wraz z parametrem $p$. Konfiguracja \texttt{spotify} jest średnio droższa niż \texttt{duolingo\_super} ze względu na typ \emph{Duo} (dokładnie 2 osoby) i odmienną strukturę cen. Udział \emph{Family} sprzyja niższym kosztom, gdy struktura grafu pozwala na większe grupy.
\end{itemize}

\section{Wnioski praktyczne i rekomendacje}
\begin{itemize}
  \item \textbf{Dobór metody do skali.} Dla bardzo małych sieci warto korzystać z ILP (kontrola jakości i walidacja). Dla średnich i dużych sieci zalecane są: zachłanny (gdy liczy się czas) lub metaheurystyki (gdy priorytetem jest koszt).
  \item \textbf{Sieci ewoluujące.} W środowiskach dynamicznych rekomendowane są metaheurystyki z warm start (projekcja poprzedniego rozwiązania), które lepiej wykorzystują ciągłość w czasie. Dla szybkich reakcji operacyjnych można łączyć zachłanny jako inicjalizację z krótkim dogrzaniem metaheurystyki.
  \item \textbf{Polityka licencji.} W sieciach z hubami opłaca się promować licencje rodzinne (większe grupy). Obecność planu \emph{Duo} zwiększa koszt średni — wskazane jest jego rozważne stosowanie (np. jako rezerwowy wariant tam, gdzie brakuje większych klastrów).
  \item \textbf{Inżynieria uruchomień.} Warto stosować cache’owanie grafów, kontrolę limitów czasu i automatyczną analizę wyników. Dla porównań między konfiguracjami rekomendowana jest normalizacja przez koszt na wierzchołek.
\end{itemize}

\section{Ograniczenia pracy}
\begin{itemize}
  \item Brak kompletnych wyników dla części \texttt{dynamic\_real}; analiza dynamiczna dotyczy obecnie grafów syntetycznych.
  \item W rozszerzeniach nie uruchomiono modelu \texttt{tree} oraz nie wykonano wariantów \texttt{duolingo\_super} poza konfiguracją bazową.
  \item Porównanie warm vs cold nie jest możliwe 1:1 dla tych samych metaheurystyk i kroków z uwagi na strukturę zebranych danych; wnioski dotyczą trajektorii oddzielnych.
  \item Strojenie parametrów metaheurystyk było ograniczone; możliwe są dalsze zyski jakościowe po kalibracji.
\end{itemize}

\section{Kierunki dalszych badań}
\begin{itemize}
  \item \textbf{Analiza dynamiczna na danych rzeczywistych.} Dokończenie \texttt{dynamic\_real}, ocena wpływu zaników i pojawień relacji w ego-sieciach.
  \item \textbf{Algorytmy online i inkrementalne.} Projekt algorytmów z gwarancjami konkurencyjności dla napływających zmian (churn), szybka reoptymalizacja lokalna.
  \item \textbf{Lepsze inicjalizacje i uczenie.} Uczenie modeli do przewidywania dobrych punktów startowych (meta-learning) oraz adaptacyjne strojenie parametrów metaheurystyk.
  \item \textbf{Rozszerzenia modelu cenowego.} Nowe typy licencji i ograniczenia (np. limity współdzielenia per okres, różne cenniki warstwowe), koszty przełączeń między planami.
  \item \textbf{Aspekty ekonomiczne i behawioralne.} Mechanizmy zachęt i stabilność koalicji, wymagania zgodności/regulacyjne, ograniczenia prywatności i polityki współdzielenia.
  \item \textbf{Granice teoretyczne.} Analiza aproksymacyjna i dolne ograniczenia dla uogólnionych wariantów (z pojemnościami), charakterystyka trudnych klas instancji.
\end{itemize}

\section{Możliwości rozwoju modelu}
\begin{itemize}
  \item Uwzględnienie zależności wielosąsiedzkich (grupy nie tylko w domknięciu $N[i]$), warianty z pośrednictwem, a także modelowanie zaufania i jakości krawędzi.
  \item Integracja ograniczeń operacyjnych: budżety, limity na liczbę zmian między krokami, koszty migracji i rozwiązania mieszane (część stała, część dynamiczna).
  \item Uogólnienie do hipergrafów lub grafów dwuwarstwowych (użytkownicy–plany), co umożliwia bogatsze relacje i reguły przypisań.
  \item Automatyczny wybór algorytmu (portfolio) w zależności od cech instancji; równoległe i rozproszone uruchomienia dla dużej skali.
\end{itemize}

\section{Zamknięcie}
Praca dostarcza spójnego modelu, przeglądu metod i solidnej warstwy eksperymentalnej dla optymalizacji licencji w sieciach społecznych. Wyniki wskazują, że w praktyce warto stosować proste heurystyki tam, gdzie kluczowy jest czas reakcji, oraz metaheurystyki z warm start w środowiskach dynamicznych, gdy priorytetem jest koszt. Dalsze prace powinny koncentrować się na pełnej analizie dynamicznej na danych rzeczywistych, rozszerzeniach modelu cenowego i rozwoju algorytmów online.
