\chapter{Podsumowanie}\label{chap:conclusion}

Praca pokazuje pełny cykl badawczy dotyczący optymalizacji kosztów licencji grupowych w sieciach społecznościowych. Najpierw sformalizowano model. Następnie porównano algorytmy deterministyczne i metaheurystyczne. Dalej przeprowadzono symulacje dynamiczne. Na końcu rozszerzono analizę o dodatkowe plany licencyjne. Uzyskano spójny obraz działania metod w wielu scenariuszach. Poniżej zebrano główne wyniki i wskazano możliwe kierunki dalszych badań.

\section{Wyniki}

\subsection{Model i metody}
Rozdziały~\ref{chap:introduction}--\ref{chap:testdata} definiują problem jako uogólnienie dominowania rzymskiego z ograniczeniami pojemności oraz różnymi typami licencji. Wykazano wysoką złożoność obliczeniową. Rozdział~\ref{chap:algorithms} prezentuje pełen zestaw metod: dokładny solver ILP, heurystyki konstrukcyjne (m.in. algorytm zachłanny, podejście przez zbiór dominujący), metaheurystyki (algorytm genetyczny, algorytm mrówkowy, przeszukiwanie tabu, wyżarzanie symulowane) oraz algorytm losowy. Wszystkie implementacje mają wspólny interfejs, co ułatwiło porównania.

\subsection{Eksperymenty statyczne}
Rozdział~\ref{chap:experiments} potwierdza hierarchię jakości: $ILP >$ algorytm mrówkowy $>$ przeszukiwanie tabu/algorytm genetyczny $>$ wyżarzanie symulowane $>$ heurystyki konstrukcyjne $>$ algorytm losowy. Różnice są istotne statystycznie (test Friedmana oraz porównania Nemenyi'ego). Grafy bezskalowe okazały się łatwiejsze do pokrycia z powodu hubów. Czas działania metaheurystyk rósł szybko, a heurystyki konstrukcyjne utrzymywały czasy rzędu milisekund. W praktyce wyróżniono trzy zakresy: dla małych grafów warto stosować ILP jako punkt odniesienia; dla średnich grafów najlepsze są metaheurystyki; dla dużych grafów albo gdy liczy się szybkość, użyteczną aproksymację daje algorytm zachłanny.

\subsection{Symulacje dynamiczne}
Rozdział~\ref{chap:dynamic} przedstawia wpływ zmian w sieci na koszt i czas. Ciepły start konsekwentnie obniżał koszt o 6--14\% (syntetyczne) i 7--12\% (realistyczne), przy czasie rebalansowania 1--3~s. Intensywniejsze mutacje zwiększały głównie czas, a mniej wpływały na koszt. Najdokładniejszy pozostawał algorytm mrówkowy, natomiast przeszukiwanie tabu i algorytm genetyczny dawały lepszy kompromis czasowy. W danych realistycznych najniższe koszty uzyskano w wariancie Spotify (mediana $\approx 0{,}41$ na węzeł).

\subsection{Rozszerzenia licencyjne}
Rozdział~\ref{chap:extensions} analizuje osiem wariantów taryf. W planach Duolingo i Roman o opłacalności decydowały rozmiar grupy i koszt planu. Przy tańszej grupie (\texttt{duolingo\_p\_2}) metaheurystyki redukowały koszt o 10--20\% względem heurystyk. Przy droższych planach (\texttt{duolingo\_p\_5}, \texttt{roman\_p\_5}) przewaga spadała do kilku procent. Dodanie planu Duo (Spotify) oraz planów Standard/Premium (Netflix) umożliwiło nowe parowania użytkowników i obniżyło koszt na węzeł (mediana 0{,}40 w Spotify wobec 0{,}50 w \texttt{duolingo\_p\_2}). W dynamice konfiguracje z planem pośrednim utrzymywały najkorzystniejszy koszt i stabilny czas, a metaheurystyki dawały 5--12\% oszczędności względem algorytmu zachłannego.

\section{Rekomendacje praktyczne}

\textbf{Dobór algorytmu.} Dla instancji do około 200 węzłów solver ILP jest najlepszym punktem odniesienia. Dla większych sieci zalecane są algorytm mrówkowy albo przeszukiwanie tabu; w środowisku dynamicznym krótsze czasy rebalansowania daje przeszukiwanie tabu.

\textbf{Strategia inicjalizacji.} Ciepły start metaheurystyk, czyli start z poprzedniego rozwiązania, obniża koszt o 6--14\% przy niewielkim koszcie czasowym. W systemach aktualizowanych częściowo powinien to być standard.

\textbf{Polityka licencyjna.} Tańsze plany rodzinne, w szczególności warianty pośrednie (np. Duo), realnie zmniejszają koszt końcowy. Zbyt wysokie $p$ w planach grupowych sprzyja licencjom indywidualnym i ogranicza zyski z optymalizacji.

\section{Kierunki dalszych badań}

\textbf{Gwarancje aproksymacji.} Warto poszukać teoretycznych ograniczeń jakości rozwiązań dla wybranych klas grafów lub modeli losowych.
\textbf{Modele stochastyczne.} Ujęcie niepewności w dostępie użytkowników i ewolucji sieci może zwiększyć odporność rozwiązań (np. podejścia dwuetapowe lub bayesowskie).
\textbf{Wielokryterialność i koszty operacyjne.} Rozszerzenie funkcji celu o sprawiedliwość, ryzyko czy koszt migracji lepiej odzwierciedli praktykę.
\textbf{Optymalizacja hiperparametrów.} Automatyczne strojenie parametrów metaheurystyk (np. techniki z rodziny AutoML) może poprawić stosunek kosztu do czasu bez ręcznego dostrajania.
\textbf{Integracja z praktyką.} Biblioteka daje podstawy do wdrożeń w systemach rekomendacji planów rodzinnych; naturalnym krokiem jest eksperyment online na rzeczywistych danych.

\section{Zakończenie}

Przedstawiony model, zestaw algorytmów oraz eksperymenty statyczne i dynamiczne pokazują, że mimo wysokiej złożoności można uzyskać rozwiązania dobrej jakości w akceptowalnym czasie. Kluczowy jest dobór metody do wielkości i dynamiki sieci oraz rozsądna polityka licencyjna. Wyniki stanowią podstawę do dalszych prac oraz do zastosowań w usługach subskrypcyjnych, gdzie modele rodzinne stają się standardem.
