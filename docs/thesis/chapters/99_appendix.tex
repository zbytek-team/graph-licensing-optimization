\appendix

\chapter{Aneks techniczny i replikowalność}

W aneksie zebrano informacje niezbędne do odtworzenia eksperymentów, szczegóły formatu danych, parametry domyślne oraz skrócone instrukcje korzystania z interfejsów CLI i skryptów analitycznych użytych w pracy.

\section{Środowisko i budowanie}
\begin{itemize}
  \item \textbf{Wersja Pythona}: 3.13 (wymagana; celowo wymuszana w projekcie).
  \item \textbf{Zarządzanie zależnościami}: \texttt{uv}. Instalacja: \texttt{make install}.
  \item \textbf{Kontrole jakości}: \texttt{make lint}, \texttt{make format}, \texttt{make typecheck}, \texttt{make test}.
  \item \textbf{Wygodne cele}: \texttt{make benchmark}, \texttt{make dynamic}, \texttt{make analyze}, \texttt{make thesis-figs}.
\end{itemize}

\section{Struktura repozytorium (wybór)}
\begin{verbatim}
src/glopt/            # biblioteka: algorytmy, model, CLI
  algorithms/         # ILP, greedy, randomized, dominating_set, GA, SA, Tabu, ACO, tree_dp
  cli/                # entry points: benchmark, dynamic, ...
  core/               # definicje Solution/LicenseType i operacje
  dynamic_simulator.py# symulacja mutacji grafu i rebalance
  license_config.py   # definicje pakietów licencyjnych
scripts/analysis/     # skrypty analityczne i generowanie wykresów
runs/                 # wyniki eksperymentów (CSV) z run_id w nazwie
results/              # produkty analizy (figury, tabele)
docs/thesis/          # pliki LaTeX i zasoby prac
Makefile              # komendy uruchomieniowe i QA
\end{verbatim}

\section{Interfejsy CLI i odtwarzanie wyników}
\begin{itemize}
  \item \texttt{benchmark}: eksperyment statyczny na grafach syntetycznych; zapis CSV do \texttt{runs/<run>/csv/}.
  \item \texttt{dynamic}: symulacja dynamiczna z mutacjami grafu (\S\,\ref{chap:dynamic}); kroki i parametry w \texttt{src/glopt/cli/dynamic.py}.
  \item \texttt{benchmark\_real}, \texttt{dynamic\_real}: warianty dla danych rzeczywistych (dynamic\_real — w toku).
  \item \texttt{analyze-csv}: ogólna analiza CSV dowolnego runu do \texttt{results/}.
\end{itemize}
Przykładowe komendy:
\begin{verbatim}
make benchmark
make dynamic
python -m scripts.analysis.analyze_csv --csv runs/<run>/csv/<run>.csv --out results/<name>
python -m scripts.analysis.dynamic_analysis --csv <dyn.csv> --tag full --out results/dynamic_analysis \
    --figdir docs/thesis/assets/figures/dynamic
python -m scripts.analysis.model_extensions --csv runs/benchmark_all/csv/<..>.csv --out results/model_extensions \
    --figdir docs/thesis/assets/figures/extensions
\end{verbatim}

\section{Format plików CSV}
Wyniki zapisywane są w postaci tabelarycznej (CSV). Pola wspólne:
\begin{itemize}
  \item \texttt{run\_id}, \texttt{algorithm}, \texttt{graph}, \texttt{n\_nodes}, \texttt{n\_edges}, \texttt{graph\_params}
  \item \texttt{license\_config}, \texttt{rep}, \texttt{seed}, \texttt{sample}, \texttt{graph\_seed}
  \item \texttt{density}, \texttt{avg\_degree}, \texttt{clustering}, \texttt{components}
  \item \texttt{total\_cost}, \texttt{cost\_per\_node}, \texttt{time\_ms}, \texttt{valid}, \texttt{issues}
  \item \texttt{groups}, \texttt{group\_size\_mean}, \texttt{group\_size\_median}, \texttt{group\_size\_p90}
  \item \texttt{license\_counts\_json}, \texttt{algo\_params\_json}, \texttt{warm\_start}, \texttt{notes}
\end{itemize}
Pola specyficzne dla dynamiki:
\begin{itemize}
  \item \texttt{step} (nr kroku), \texttt{mutation\_params\_json}, \texttt{mutations} (opis zmian w danym kroku).
\end{itemize}

\section{Konfiguracje licencji}
Definicje w \texttt{src/glopt/license\_config.py}. Najważniejsze zestawy:
\begin{itemize}
  \item \texttt{duolingo\_super}: Individual (13.99; 1–1), Family (29.17; 2–6)
  \item \texttt{spotify}: Individual (23.99; 1–1), Duo (30.99; 2–2), Family (37.99; 2–6)
  \item \texttt{roman\_domination}: Solo (1.0; 1–1), Group (2.0; 2–99999)
  \item \texttt{roman\_p\_x}: Solo (1.0; 1–1), Group ($p$; 2–99999) — skan po $p\in\{1.5,2.0,2.5,3.0\}$
  \item \texttt{duolingo\_p\_x}: Individual (1.0; 1–1), Family ($p$; 2–6)
\end{itemize}

\section{Parametry algorytmów (domyślne)}
\begin{itemize}
  \item \textbf{ILP}: solver CBC, opcjonalny \texttt{time\_limit}.
  \item \textbf{Greedy}: bez parametrów, porządkowanie po stopniu, wybór najbardziej opłacalnej licencji.
  \item \textbf{Randomized}: losowy przydział zgodny z pojemnościami; opcjonalny \texttt{seed}.
  \item \textbf{Dominating Set}: heurystyka dominująca z doborem grup po koszt/rozmiar.
  \item \textbf{Genetic}: \texttt{population\_size}=30, \texttt{generations}=40, \texttt{elite\_fraction}=0.2, \texttt{crossover\_rate}=0.6.
  \item \textbf{Simulated Annealing}: \texttt{T0}=100, \texttt{cooling}=0.995, \texttt{T\_min}=0.001, \texttt{max\_iterations}=20\,000, \texttt{max\_stall}=2\,000.
  \item \textbf{Tabu Search}: \texttt{max\_iterations}=1000, \texttt{tabu\_tenure}=20, \texttt{neighbors\_per\_iter}=10.
  \item \textbf{Ant Colony}: \texttt{alpha}=1.0, \texttt{beta}=2.0, \texttt{evaporation}=0.5, \texttt{q0}=0.9, \texttt{num\_ants}=20, \texttt{max\_iterations}=100.
\end{itemize}
Metaheurystyki wspierają \emph{warm start} (przekazanie rozwiązania początkowego) w trybach benchmark i dynamic.

\section{Symulacja dynamiczna: parametry}
Parametry w \texttt{src/glopt/cli/dynamic.py} i \texttt{dynamic\_simulator.py}:
\begin{itemize}
  \item Prawdopodobieństwa mutacji (przykładowe): \texttt{add\_nodes}=0.06, \texttt{remove\_nodes}=0.04, \texttt{add\_edges}=0.18, \texttt{remove\_edges}=0.12.
  \item Limity zmian na krok skalowane z rozmiarem grafu: maks. dodawanych/usuwanych węzłów i krawędzi proporcjonalnie do \(n\) i \(|E|\).
  \item Domyślnie \texttt{NUM\_STEPS}=50; w serii \texttt{dynamic\_20} — 100.
  \item Agregowane metryki w każdym kroku: koszt, czas, statystyki rozmiarów grup, opis mutacji.
\end{itemize}

\section{Nazewnictwo runów i ścieżki}
Id runu ma postać \texttt{YYYYMMDD\_HHMMSS\_<sufiks>}, np. \texttt{20250907\_014311\_dynamic}. Wyniki znajdują się w \texttt{runs/<id>/csv/<id>.csv}. Figury eksportowane do pracy: \texttt{docs/thesis/assets/figures/...}. Produkty analiz: \texttt{results/}.

\section{Skrypty analityczne}
\begin{itemize}
  \item \texttt{analyze\_csv.py}: przegląd statyczny (koszt/czas vs \(n\), Pareto, heatmapy, profile wydajności).
  \item \texttt{dynamic\_analysis.py}: trajektorie kosztu/czasu, intensywność mutacji, (opc.) przewaga warm.
  \item \texttt{model\_extensions.py}: porównania konfiguracji (roman\_p\_x, spotify), krzywe kosztu vs \(p\), miks licencji.
  \item \texttt{export\_thesis\_figs.py}: eksport wybranych figur do katalogu pracy.
\end{itemize}

\section{Uwagi replikacyjne}
\begin{itemize}
  \item Generatory grafów są deterministyczne względem \texttt{seed}; cache grafów przyspiesza uruchomienia (\texttt{data/graphs\_cache}).
  \item Ustawienie zmiennej \texttt{ANALYZE\_PDF=1} dodaje wersje PDF figur obok PNG.
  \item Dla bardzo dużych grafów ograniczono konfiguracje licencji i algorytmy zgodnie z rozdziałami eksperymentalnymi (limity czasu).
\end{itemize}
