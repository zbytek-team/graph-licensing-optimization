\chapter{Analiza dynamicznej wersji problemu}

\section{Wprowadzenie do dynamicznego modelu}

W praktycznych zastosowaniach sieci społecznościowe są strukturami dynamicznymi, podlegającymi częstym zmianom, takim jak dodawanie lub usuwanie użytkowników, a także tworzenie nowych lub zrywanie istniejących relacji. Z tego względu naturalnym rozwinięciem problemu optymalizacji zakupu licencji oprogramowania jest rozważenie modelu dynamicznego, w którym graf społecznościowy ewoluuje w czasie.

W niniejszym rozdziale przedstawiono analizę dynamicznej wersji problemu, omówiono rodzaje zmian, jakim może podlegać graf, oraz przeprowadzono analizę skuteczności algorytmów adaptacyjnych.

\section{Rodzaje zmian dynamicznych w grafach}

W dynamicznym modelu grafu \(G = (V, E)\) możemy wyróżnić następujące typy zmian:

\begin{itemize}
    \item \textbf{Dodawanie wierzchołków} – nowi użytkownicy dołączający do społeczności.
    \item \textbf{Usuwanie wierzchołków} – użytkownicy opuszczający społeczność lub rezygnujący z subskrypcji.
    \item \textbf{Dodawanie krawędzi} – powstawanie nowych relacji znajomości.
    \item \textbf{Usuwanie krawędzi} – zrywanie istniejących relacji znajomości.
\end{itemize}

Każda z powyższych operacji wymusza konieczność aktualizacji konfiguracji zakupionych licencji, aby nadal spełniały one wszystkie wymagania dotyczące dominowania oraz ograniczeń rozmiaru grup.

\section{Adaptacja algorytmów do wersji dynamicznej}

W modelu dynamicznym kluczowe jest szybkie i efektywne aktualizowanie rozwiązania, zamiast jego ponownego obliczania od podstaw po każdej zmianie. Adaptację algorytmów można realizować za pomocą:

\subsection{Algorytmy lokalnego przeszukiwania (Local Search, Iterated Local Search – ILS)}

\begin{itemize}
    \item Rozwiązanie początkowe (przed zmianą grafu) jest wykorzystywane jako punkt startowy.
    \item Po dokonaniu zmian w grafie stosuje się lokalne operacje (dodawanie/usuwanie licencji indywidualnych/grupowych) do ponownego osiągnięcia stanu optymalnego lub bliskiego optymalnemu.
\end{itemize}

\subsection{Algorytmy genetyczne adaptacyjne}

\begin{itemize}
    \item Populacja rozwiązań z poprzedniej iteracji jest wykorzystywana jako punkt startowy dla kolejnej.
    \item Mutacje lub krzyżowania rozwiązań uwzględniają zmiany struktury grafu (np. usuwanie nieprawidłowych grup po zerwaniu relacji).
\end{itemize}

% (Usunięto rozważania RL — niewykorzystane w implementacji.)

\section{Eksperymenty z dynamicznymi grafami}

Eksperymenty realizujemy jako sekwencję kroków z mutacjami grafu (dodania/usunięcia wierzchołków i krawędzi), porównując warianty cold- i warm-start dla wybranych metaheurystyk (GA/SA/TS/ACO) względem rozwiązań przeliczanych od zera. Miary raportujemy analogicznie jak w rozdz.\,6 (czas/krok, koszt, zmiany grup). Wyniki wskazują na istotne korzyści warm-startu względem cold-startu przy zachowaniu jakości.

\section{Jak uruchomić wersję dynamiczną w praktyce}
W implementacji dostępne są dwa skrypty CLI: \texttt{dynamic.py} (syntetyczne) oraz \texttt{dynamic\_real.py} (ego-sieci). Skrypty logują parametry, a następnie w pętli po krokach stosują mutacje grafu i uruchamiają algorytmy w dwóch wariantach: cold- oraz warm-start. Warm-start przekazuje rozwiązanie z poprzedniego kroku jako punkt wyjścia, co przyspiesza obliczenia przy porównywalnej jakości.

Wyniki dynamiczne zapisuję dokładnie tak jak wyniki statyczne - CSV i potem wykresy z analizy. W pracy pokazuję w rozdziale eksperymenty wykresy czasu i kosztu na kolejnych krokach oraz porównania warm vs cold. To daje jasny obraz przyspieszenia dzięki podgrzewaniu rozwiązania.

\section{Analiza stabilności rozwiązań dynamicznych}

Analizę stabilności prowadzimy, mierząc liczbę zmian przydziałów w kolejnych krokach. W praktyce metaheurystyki z warm-startem ograniczają liczbę zmian względem przeliczeń od zera, co sprzyja stabilności konfiguracji.

\section{Podsumowanie rozdziału}

W tym rozdziale przeanalizowano problem optymalizacji licencji w wersji dynamicznej. Przedstawiono strategie adaptacyjne, umożliwiające szybką i efektywną aktualizację rozwiązań w odpowiedzi na zmiany zachodzące w grafie społecznościowym. Eksperymenty pokazały, że podejścia wykorzystujące lokalne przeszukiwanie oraz metaheurystyki (GA/SA/TS/ACO) pozwalają na skuteczną adaptację przy zachowaniu niskiego wzrostu kosztu i krótkiego czasu reakcji na zmiany w strukturze sieci.

W kolejnym rozdziale przedstawione zostaną rozszerzenia modelu związane z różnymi wariantami polityk cenowych oraz typami planów subskrypcyjnych.
