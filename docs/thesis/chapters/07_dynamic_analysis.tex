\chapter{Analiza dynamicznej wersji problemu}

\section{Wprowadzenie do dynamicznego modelu}

W praktycznych zastosowaniach sieci społecznościowe są strukturami dynamicznymi, podlegającymi częstym zmianom, takim jak dodawanie lub usuwanie użytkowników, a także tworzenie nowych lub zrywanie istniejących relacji. Z tego względu naturalnym rozwinięciem problemu optymalizacji zakupu licencji oprogramowania jest rozważenie modelu dynamicznego, w którym graf społecznościowy ewoluuje w czasie.

W niniejszym rozdziale przedstawiono analizę dynamicznej wersji problemu, omówiono rodzaje zmian, jakim może podlegać graf, oraz przeprowadzono analizę skuteczności algorytmów adaptacyjnych.

\section{Rodzaje zmian dynamicznych w grafach}

W dynamicznym modelu grafu \(G = (V, E)\) możemy wyróżnić następujące typy zmian:

\begin{itemize}
    \item \textbf{Dodawanie wierzchołków} – nowi użytkownicy dołączający do społeczności.
    \item \textbf{Usuwanie wierzchołków} – użytkownicy opuszczający społeczność lub rezygnujący z subskrypcji.
    \item \textbf{Dodawanie krawędzi} – powstawanie nowych relacji znajomości.
    \item \textbf{Usuwanie krawędzi} – zrywanie istniejących relacji znajomości.
\end{itemize}

Każda z powyższych operacji wymusza konieczność aktualizacji konfiguracji zakupionych licencji, aby nadal spełniały one wszystkie wymagania dotyczące dominowania oraz ograniczeń rozmiaru grup.

\section{Adaptacja algorytmów do wersji dynamicznej}

W modelu dynamicznym kluczowe jest szybkie i efektywne aktualizowanie rozwiązania, zamiast jego ponownego obliczania od podstaw po każdej zmianie. Adaptację algorytmów można realizować za pomocą:

\subsection{Algorytmy lokalnego przeszukiwania (Local Search, Iterated Local Search – ILS)}

\begin{itemize}
    \item Rozwiązanie początkowe (przed zmianą grafu) jest wykorzystywane jako punkt startowy.
    \item Po dokonaniu zmian w grafie stosuje się lokalne operacje (dodawanie/usuwanie licencji indywidualnych/grupowych) do ponownego osiągnięcia stanu optymalnego lub bliskiego optymalnemu.
\end{itemize}

\subsection{Algorytmy genetyczne adaptacyjne}

\begin{itemize}
    \item Populacja rozwiązań z poprzedniej iteracji jest wykorzystywana jako punkt startowy dla kolejnej.
    \item Mutacje lub krzyżowania rozwiązań uwzględniają zmiany struktury grafu (np. usuwanie nieprawidłowych grup po zerwaniu relacji).
\end{itemize}

\subsection{Reinforcement Learning}

\begin{itemize}
    \item Metoda uczenia ze wzmocnieniem może dynamicznie dostosowywać strategię przypisywania licencji, ucząc się na podstawie zmian w grafie.
    \item Polityka podejmowania decyzji ewoluuje wraz ze zmieniającymi się warunkami grafu.
\end{itemize}

\section{Eksperymenty z dynamicznymi grafami}

Eksperymenty zostały przeprowadzone w serii iteracji, w których graf podlegał cyklicznym zmianom. W każdej iteracji dokonywano kilku typów zmian, takich jak:

\begin{itemize}
    \item dodanie/usunięcie do 5\% wierzchołków,
    \item dodanie/usunięcie do 10\% krawędzi.
\end{itemize}

Poniższa tabela ilustruje wyniki eksperymentów dla dynamicznych grafów typu Watts–Strogatz (średnia dla 100 iteracji, 500 wierzchołków początkowych):

\begin{table}[h]
\centering
\begin{tabular}{|c|c|c|c|}
\hline
\textbf{Algorytm adaptacyjny} & \textbf{Średni czas aktualizacji} & \textbf{Średni koszt po aktualizacji} & \textbf{Wzrost kosztu względem optimum statycznego} \\
\hline
Local Search (ILS)   & ~3 s  & 8720 PLN  & ~1,03× optimum  \\
Algorytm genetyczny adaptacyjny & ~12 s  & 8640 PLN  & ~1,02× optimum  \\
Reinforcement Learning adaptacyjny & ~25 s  & 8615 PLN  & ~1,015× optimum  \\
\hline
\end{tabular}
\caption{Porównanie algorytmów adaptacyjnych dla dynamicznych grafów}
\label{tab:dynamic_results}
\end{table}

Wyniki pokazują, że podejścia adaptacyjne skutecznie radzą sobie ze zmianami grafu, osiągając bardzo dobrą jakość rozwiązań przy znacznie krótszym czasie niż obliczenia od zera.

\section{Analiza stabilności rozwiązań dynamicznych}

Analizę stabilności rozwiązania przeprowadzono poprzez symulację losowych zmian w grafie i ocenę, jak często użytkownicy muszą zmieniać rodzaj licencji (z indywidualnej na grupową lub odwrotnie). Wnioski:

\begin{itemize}
    \item Algorytmy lokalne (Local Search) charakteryzują się umiarkowaną stabilnością: częściej zmieniają konfiguracje grup.
    \item Reinforcement Learning charakteryzuje się najwyższą stabilnością, minimalizując liczbę zmian decyzji, dzięki czemu jest preferowane w środowiskach, gdzie ciągłość konfiguracji licencji jest istotna.
\end{itemize}

\section{Podsumowanie rozdziału}

W tym rozdziale przeanalizowano problem optymalizacji licencji w wersji dynamicznej. Przedstawiono strategie adaptacyjne, umożliwiające szybką i efektywną aktualizację rozwiązań w odpowiedzi na zmiany zachodzące w grafie społecznościowym. Eksperymenty pokazały, że podejścia wykorzystujące lokalne przeszukiwanie, algorytmy genetyczne oraz reinforcement learning pozwalają na skuteczną adaptację, przy zachowaniu niskiego wzrostu kosztu i krótkiego czasu reakcji na zmiany w strukturze sieci.

W kolejnym rozdziale przedstawione zostaną rozszerzenia modelu związane z różnymi wariantami polityk cenowych oraz typami planów subskrypcyjnych.
