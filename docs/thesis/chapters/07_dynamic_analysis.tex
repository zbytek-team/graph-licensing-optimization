\chapter{Symulacja dynamiczna}\label{chap:dynamic}

\section{Założenia i konfiguracja}

Eksperymenty dynamiczne wykonano w tym samym środowisku obliczeniowym co testy statyczne. Każda symulacja obejmuje 30 kroków, a po każdej mutacji sieci algorytmy otrzymują maksymalnie 45~s na ponowne zbilansowanie licencji. Analizowano dwie konfiguracje licencyjne -- Duolingo Super oraz dominowanie rzymskie.

\subsection{Poziomy intensywności mutacji}

Warianty \emph{low}, \emph{med} oraz \emph{high} różnią się prawdopodobieństwami modyfikacji wierzchołków i krawędzi (tab.~\ref{tab:dyn-mutation-levels}).

\begin{table}[H]
  \centering
  \caption{Parametry intensywności mutacji w symulacji dynamicznej.}
  \label{tab:dyn-mutation-levels}
  \begin{tabular}{lcccc}
    \toprule
    \textbf{Poziom} & \textbf{Dodawanie węzłów} & \textbf{Usuwanie węzłów} & \textbf{Dodawanie krawędzi} & \textbf{Usuwanie krawędzi} \\
    \midrule
    niski           & 0.02                      & 0.01                     & 0.06                        & 0.04                       \\
    średni          & 0.06                      & 0.04                     & 0.18                        & 0.12                       \\
    wysoki          & 0.12                      & 0.08                     & 0.30                        & 0.20                       \\
  \end{tabular}
\end{table}

\subsection{Scenariusze realistyczne}

Dodatkowo zbadano trzy bardziej realistyczne profile ewolucji sieci. Wszystkie operują na tych samych limitach liczby dodawanych/usuwanych elementów co warianty syntetyczne, różnią się jednak mechanizmem wyboru sąsiedztwa. Wariant \emph{pref\_triadic} tworzy nowe wierzchołki z preferencyjnym przyłączaniem (stopień +1) i zamyka trójkąty w sąsiedztwie bieżących wierzchołków. Wariant \emph{pref\_pref} stosuje preferencyjne przyłączanie zarówno do węzłów, jak i krawędzi. Wreszcie \emph{rand\_rewire} łączy losowe dodawanie węzłów z przekształcaniem istniejących krawędzi w stylu Wattsa--Strogatza, gdzie losowa zamiana końców krawędzi zachodzi z prawdopodobieństwem 0,1 na operację.

\section{Algorytm zachłanny}

W tym wariancie algorytm zachłanny w każdym kroku symulacji buduje rozwiązanie całkowicie od zera. Stanowi to dolną granicę narzutu czasowego dla metod bez pamięci oraz bazowy punkt odniesienia jakości, do którego porównujemy bardziej zaawansowane metaheurystyki korzystające z rozwiązań z poprzedniego kroku.

\subsection{Zestawienie dla wszystkich mutacji}
Tabela~\ref{tab:greedy-cold-summary} zestawia średni koszt na węzeł i średni czas na krok dla sześciu badanych profili mutacji: trzech syntetycznych oraz trzech realistycznych. Wszystkie czasy mieszczą się w przedziale 0.5--1.8 ms na krok, a średni koszt na węzeł oscyluje wokół 0.46--0.48.

\begin{table}[H]
  \centering
  \caption{Algorytm zachłanny: średni koszt na węzeł oraz średni czas na krok dla wszystkich wariantów mutacji.}
  \label{tab:greedy-cold-summary}
  \begin{tabular}{lcc}
    oprule
    extbf{Metoda mutacji} & \textbf{Koszt/węzeł (mean)} & \textbf{Średni czas [s]} \\
    \midrule
    high                  & 0.48009                     & 0.00159                  \\
    low                   & 0.47983                     & 0.00165                  \\
    med                   & 0.47491                     & 0.00181                  \\
    pref\_pref            & 0.46371                     & 0.000835                 \\
    pref\_triadic         & 0.46787                     & 0.000535                 \\
    rand\_rewire          & 0.47556                     & 0.000833                 \\
    \bottomrule
  \end{tabular}
\end{table}

Algorytm zachłanny jest bardzo szybki (sub-milisekundowy do \SI{1.8}{\ms}), co potwierdza jego przydatność jako lekki baseline czasowy w środowisku dynamicznym. Warianty realistyczne sprzyjają nieco niższym kosztom: \texttt{pref\_pref} osiąga najniższy średni koszt (0.464), a \texttt{pref\_triadic} jest jednocześnie najszybszy (\SI{0.000535}{\s}). Wariant \texttt{rand\_rewire} jest trudniejszy (0.476), ale pozostaje bardzo szybki czasowo.

Różnice kosztu dla mutacji syntetycznych są niewielkie (0.475--0.480), natomiast czasy są wyższe niż w profilach realistycznych, szczególnie dla \texttt{med/high}. Wskazuje to, że bardziej lokalne, realistyczne przekształcenia struktury grafu są łatwiejsze do obsłużenia. Mediany czasów są niższe niż średnie (np. \texttt{pref\_pref}: mediana \SI{0.00058}{\s} vs średnia \SI{0.00083}{\s}, \emph{n}=930; \texttt{pref\_triadic}: \SI{0.00046}{\s} vs \SI{0.00053}{\s}, \emph{n}=744), co sugeruje długi ogon rzadkich, nieco wolniejszych kroków. Analogiczne zjawisko obserwujemy w mutacjach syntetycznych (\emph{n}=1116 na wariant).

\section{Wyniki na mutacjach syntetycznych}
\subsection{Metaheurystyki}

Tabela~\ref{tab:dyn-synth-warm} przedstawia zbiorcze wyniki dla różnych metod mutacji. Średni koszt na węzeł jest bardzo zbliżony dla wszystkich poziomów intensywności, z jedynie nieznacznym wzrostem dla wariantu \texttt{high}. Sugeruje to, że algorytmy są w stanie skutecznie adaptować się do zmian w topologii sieci.

Średni czas wykonania rośnie wraz z intensywnością mutacji. Wariant \texttt{low} jest najszybszy, podczas gdy \texttt{high} wymaga najwięcej czasu na ponowne zbilansowanie. Jest to naturalna konsekwencja faktu, że większa liczba modyfikacji grafu (dodawanie/usuwanie węzłów i krawędzi) stanowi większe wyzwanie obliczeniowe dla algorytmów optymalizacyjnych. Mimo to, różnice w czasach nie są drastyczne, co świadczy o dobrej skalowalności zastosowanych metod.

\begin{table}[H]
  \centering
  \caption{Wyniki dla różnych metod mutacji.}
  \label{tab:dyn-synth-warm}
  \begin{tabular}{lcc}
    \toprule
    \textbf{Metoda mutacji} & \textbf{Średni koszt} & \textbf{Średni czas [s]} \\
    \midrule
    high                    & 0.4893                & 3.169                    \\
    med                     & 0.4850                & 3.075                    \\
    low                     & 0.4852                & 2.878                    \\
    \bottomrule
  \end{tabular}
\end{table}

\subsection{Profil kosztu i czasu w czasie}
Jako przykład ewolucji kosztu i czasu w krokach symulacji wybrano algorytm genetyczny. Rysunki~\ref{fig:dyn-synth-genetic-cost} i~\ref{fig:dyn-synth-genetic-time} przedstawiają odpowiednio przebieg kosztu na węzeł oraz czasu wykonania w zależności od kroku symulacji dla różnych poziomów intensywności mutacji.

Dla wariantu \texttt{high} można zaobserwować niewielkie, lecz zauważalne wahania średniego kosztu na węzeł, które nie są tak widoczne w przypadku wariantów \texttt{low} i \texttt{med}. Jeśli chodzi o czasy wykonania, dla każdego z trzech wariantów widoczne są zbliżone wahania w trakcie trwania symulacji.

\begin{figure}[H]
  \centering
  \includegraphics[width=0.32\linewidth]{assets/figures/dynamic/synthetic/synthetic_algorytm_genetyczny_cost_over_steps_low.pdf}
  \includegraphics[width=0.32\linewidth]{assets/figures/dynamic/synthetic/synthetic_algorytm_genetyczny_cost_over_steps_med.pdf}
  \includegraphics[width=0.32\linewidth]{assets/figures/dynamic/synthetic/synthetic_algorytm_genetyczny_cost_over_steps_high.pdf}
  \caption{Algorytm genetyczny -- koszt na węzeł w funkcji kroku (warianty low/med/high).}
  \label{fig:dyn-synth-genetic-cost}
\end{figure}

\begin{figure}[H]
  \centering
  \includegraphics[width=0.32\linewidth]{assets/figures/dynamic/synthetic/synthetic_algorytm_genetyczny_time_over_steps_low.pdf}
  \includegraphics[width=0.32\linewidth]{assets/figures/dynamic/synthetic/synthetic_algorytm_genetyczny_time_over_steps_med.pdf}
  \includegraphics[width=0.32\linewidth]{assets/figures/dynamic/synthetic/synthetic_algorytm_genetyczny_time_over_steps_high.pdf}
  \caption{Algorytm genetyczny -- czas wykonania w funkcji kroku (warianty low/med/high).}
  \label{fig:dyn-synth-genetic-time}
\end{figure}

\section{Wyniki na mutacjach realistycznych}

Tabela~\ref{tab:dyn-real-warm} przedstawia zbiorcze wyniki dla scenariuszy realistycznych. Wariant \texttt{pref\_triadic} wyróżnia się najkrótszym średnim czasem wykonania (poniżej 1 sekundy), przy zachowaniu kosztu na poziomie zbliżonym do wariantu \texttt{pref\_pref}. Scenariusz \texttt{rand\_rewire} okazał się najtrudniejszy -- charakteryzuje się zarówno najwyższym średnim kosztem, jak i najdłuższym czasem przetwarzania.

\begin{table}[H]
  \centering
  \caption{Wyniki dla różnych metod mutacji w scenariuszach realistycznych.}
  \label{tab:dyn-real-warm}
  \begin{tabular}{lcc}
    \toprule
    \textbf{Metoda mutacji} & \textbf{Średni koszt} & \textbf{Średni czas [s]} \\
    \midrule
    pref\_pref              & 0.4699                & 1.743                    \\
    pref\_triadic           & 0.4704                & 0.896                    \\
    rand\_rewire            & 0.4855                & 1.908                    \\
    \bottomrule
  \end{tabular}
\end{table}

\subsection{Wybrane algorytmy i metody mutacji}
Poniżej wybrano kilka par algorytmów i metod mutacji, które pokazują różne kompromisy. Dla porównania uwzględniono również dwie mutacje syntetyczne. Nie są to zawsze najlepsze jakościowo konfiguracje, ale dobrze ilustrują różne scenariusze.

\begin{table}[H]
  \centering
  \caption{Wybrane pary algorytmów i metod mutacji (różne kompromisy).}
  \label{tab:dyn-synth-selected-best}
  \begin{tabular}{llcc}
    \toprule
    \textbf{Algorytm}     & \textbf{Metoda} & \textbf{Koszt/węzeł} & \textbf{Śr. czas [s]} \\
    \midrule
    Solver ILP            & pref\_triadic   & 0.362                & 1.525                 \\
    Solver ILP            & rand\_rewire    & 0.390                & 2.553                 \\
    Algorytm genetyczny   & pref\_triadic   & 0.409                & 0.615                 \\
    Algorytm genetyczny   & high            & 0.429                & 3.269                 \\
    Przeszukiwanie tabu   & pref\_triadic   & 0.413                & 1.470                 \\
    Przeszukiwanie tabu   & high            & 0.447                & 6.454                 \\
    Algorytm mrówkowy     & pref\_pref      & 0.417                & 6.906                 \\
    Algorytm mrówkowy     & pref\_triadic   & 0.424                & 3.002                 \\
    Wyżarzanie symulowane & pref\_triadic   & 0.460                & 0.555                 \\
    Algorytm zachłanny    & pref\_pref      & 0.464                & 0.001                 \\
    Zbiór dominujący      & pref\_triadic   & 0.457                & 0.005                 \\
    Algorytm losowy       & pref\_pref      & 0.754                & 0.001                 \\
    \bottomrule
  \end{tabular}
\end{table}

Solver ILP osiąga najniższe koszty, ale wymaga około 1.5--2.6 sekundy na krok. Algorytm genetyczny dobrze sprawdza się przy zmianach klastrowych (\texttt{pref\_triadic}), oferując niski koszt i czas około 0.6 sekundy. Jednak przy intensywnych mutacjach (\texttt{high}) staje się wyraźnie wolniejszy i droższy. Z kolei wyżarzanie symulowane jest szybkie (około pół sekundy) i stabilne, ale jakościowo ustępuje bardziej zaawansowanym metodom.

Przeszukiwanie tabu korzysta z lokalności zmian, osiągając sensowny koszt i czas około 1.5 sekundy przy \texttt{pref\_triadic}. Jednak przy intensywnych mutacjach (\texttt{high}) czas wzrasta do około 6.5 sekundy, a koszt również rośnie. Algorytm mrówkowy zapewnia bardzo dobrą jakość przy \texttt{pref\_pref}, ale działa najwolniej. Przejście na \texttt{pref\_triadic} prawie podwaja szybkość kosztem niewielkiego pogorszenia jakości.

Heurystyki szybkie, takie jak algorytm zachłanny (około 1 ms) i zbiór dominujący (około 5 ms), są bardzo efektywne. Zbiór dominujący zwykle osiąga niższy koszt niż zachłanny, co czyni go dobrym wyborem przy ograniczeniach czasowych. Mutacje klastrowe (\texttt{pref\_triadic}, \texttt{pref\_pref}) pomagają wszystkim algorytmom utrzymać niski koszt i krótszy czas. Natomiast \texttt{rand\_rewire} i wariant \texttt{high} zwiększają zarówno koszt, jak i czas.

\subsection{Ewolucja kosztów w czasie}
Pełne przebiegi dla algorytmu genetycznego pokazano na rys.~\ref{fig:dyn-real-genetic-cost}--\ref{fig:dyn-real-genetic-time}. Łączenie preferencyjnego przyłączania z triadycznym domykaniem sprzyja utrzymaniu najniższych kosztów, natomiast wariant z losowym przełączaniem krawędzi prowadzi do wolniejszej stabilizacji. Analizując koszt na węzeł (Rys. \ref{fig:dyn-real-genetic-cost}), można zauważyć, że dla każdego z typów mutacji przebiega on podobnie, oscylując wokół zbliżonego poziomu.

Z kolei, patrząc na czas wykonania (Rys. \ref{fig:dyn-real-genetic-time}), widać, że dla wariantu \texttt{rand\_rewire} występują największe wahania, co sugeruje większą niestabilność w procesie optymalizacji. Warianty \texttt{pref\_triadic} i \texttt{pref\_pref} charakteryzują się bardziej stabilnym czasem wykonania.

\begin{figure}[H]
  \centering
  \includegraphics[width=0.32\linewidth]{assets/figures/dynamic/real/real_algorytm_genetyczny_cost_over_steps_pref_triadic.pdf}
  \includegraphics[width=0.32\linewidth]{assets/figures/dynamic/real/real_algorytm_genetyczny_cost_over_steps_pref_pref.pdf}
  \includegraphics[width=0.32\linewidth]{assets/figures/dynamic/real/real_algorytm_genetyczny_cost_over_steps_rand_rewire.pdf}
  \caption{Algorytm genetyczny -- koszt na węzeł w wariantach realistycznych.}
  \label{fig:dyn-real-genetic-cost}
\end{figure}

\begin{figure}[H]
  \centering
  \includegraphics[width=0.32\linewidth]{assets/figures/dynamic/real/real_algorytm_genetyczny_time_over_steps_pref_triadic.pdf}
  \includegraphics[width=0.32\linewidth]{assets/figures/dynamic/real/real_algorytm_genetyczny_time_over_steps_pref_pref.pdf}
  \includegraphics[width=0.32\linewidth]{assets/figures/dynamic/real/real_algorytm_genetyczny_time_over_steps_rand_rewire.pdf}
  \caption{Algorytm genetyczny -- czas wykonania w wariantach realistycznych.}
  \label{fig:dyn-real-genetic-time}
\end{figure}

\subsection{Wnioski}
Analiza dynamiczna pokazała, że metaheurystyki radzą sobie z adaptacją do zmian w sieci i utrzymaniem jakości rozwiązań bez potrzeby liczenia wszystkiego od nowa.

Zwiększenie intensywności zmian, szczególnie w testach syntetycznych, powodowało wydłużenie czasu potrzebnego na zrównoważenie. Algorytmy takie jak przeszukiwanie tabu miały też tendencję do wzrostu kosztu przy bardzo dużych zmianach.

Metody zmian oparte na preferencyjnym dołączaniu i zamykaniu trójkątów (\texttt{pref\_triadic}) tworzyły struktury, które łatwiej było optymalizować. Przekładało się to na niższy koszt i krótszy czas działania większości algorytmów.

Solver ILP nadal dawał najlepsze wyniki pod względem jakości, ale działał dość długo. Z kolei proste metody, jak algorytm zachłanny, działały bardzo szybko, ale dawały gorsze wyniki. Metaheurystyki były czymś pośrodku, oferując dobry balans między jakością a czasem.

Reasumując, symulacje dynamiczne potwierdziły, że adaptacyjne zarządzanie licencjami za pomocą metaheurystyk ma sens. Pozwalają one na utrzymanie stabilnego kosztu w zmieniającej się sieci, przy akceptowalnym kompromisie między jakością a czasem.
