\chapter{Rozszerzenia modelu i dodatkowe konfiguracje}\label{chap:extensions}

Przyjęty w poprzednich rozdziałach model zakładał dwa rodzaje licencji (\texttt{duolingo\_super} i \texttt{roman\_domination}). W praktyce istnieje wiele wariantów subskrypcji o różnych cenach i limitach grupowych. Niniejszy rozdział opisuje rozszerzenia modelu uwzględniające:
\begin{itemize}
  \item różne poziomy ceny licencji grupowej w wariancie \texttt{roman\_domination} (parametr $p$),
  \item dodatkowy plan Duo występujący w ofercie serwisu Spotify,
  \item trzy warianty subskrypcji platformy Netflix.
\end{itemize}

Eksperymenty obejmujące te warianty zostały zaplanowane, lecz na moment powstawania rozdziału nie są jeszcze dostępne pliki wynikowe z przebiegu testów. Dlatego część sekcji dotyczących analizy i wyników oznaczono jako TODO.

\section{Warianty roman\_p\_x}

W bazowej konfiguracji \texttt{roman\_domination} występują dwa typy licencji:
\begin{itemize}
  \item \textbf{Solo} -- koszt 1,0 jednostki za jedną osobę,
  \item \textbf{Group} -- koszt 2,0 jednostki niezależnie od liczby osób (nieograniczona pojemność).
\end{itemize}

Warianty \texttt{roman\_p\_x} wprowadzają parametr $p > 1$, określający koszt licencji grupowej. W kodzie systemu wartość $p$ jest definiowana w nazwie konfiguracji: \texttt{roman\_p\_1\_5}, \texttt{roman\_p\_2\_5}, \texttt{roman\_p\_3\_0} odpowiadają odpowiednio kosztom 1,5; 2,5; 3,0 dla licencji grupowej. Pozostałe parametry (nielimitowana pojemność, koszt licencji Solo = 1,0) pozostają niezmienione.

Wzrost parametru $p$ powoduje większą relatywną opłacalność licencji Solo; w związku z tym oczekuje się, że dla wyższych $p$ algorytmy będą tworzyć mniejsze grupy, a w skrajnym przypadku (duży $p$) możliwy jest powrót do wariantu z dominacją klasyczną. Planowane analizy obejmują:
\begin{itemize}
  \item zależność kosztu na wierzchołek od wielkości grafu dla poszczególnych $p$,
  \item porównanie rozkładu wielkości grup przy różnych wartościach $p$,
  \item badanie udziału licencji Solo vs. Group w rozwiązaniach.
\end{itemize}

\emph{TODO: rysunki i tabele opisujące te analizy zostaną dodane w finalnej wersji pracy.}

\section{Konfiguracja Spotify}

W usłudze Spotify występują trzy rodzaje licencji:
\begin{itemize}
  \item \textbf{Individual} -- cena 23,99 (jedna osoba, pojemność 1),
  \item \textbf{Duo} -- cena 30,99 (dokładnie 2 użytkowników, pojemność 2),
  \item \textbf{Family} -- cena 37,99 (od 2 do 6 osób).
\end{itemize}

W planie Duo dwóch abonentów dzieli wspólny plan, zachowując oddzielne konta premium. Wymagane jest, aby obie osoby mieszkały pod tym samym adresem. Dostępność licencji Duo wprowadza nową strategię grupowania: optymalny algorytm może preferować tworzenie par (pojemność 2) zamiast większych grup (pojemność 6).

\emph{TODO: eksperymenty obejmą porównanie kosztów i czasów dla tej konfiguracji z poprzednimi wariantami. Rysunki (koszt na wierzchołek i czas vs $n$, fronty Pareto) zostaną dołączone po uzyskaniu wyników.}

\section{Konfiguracja Netflix}

Subskrypcja platformy Netflix została uproszczona do trzech planów:
\begin{itemize}
  \item \textbf{StdWithAds} -- cena 7,99 (dla 1 użytkownika),
  \item \textbf{Standard} -- cena 17,99 (dla 2 użytkowników, pojemność 2),
  \item \textbf{Premium} -- cena 24,99 (dla 2--4 użytkowników, pojemność 4).
\end{itemize}

W odróżnieniu od Spotify, Netflix nie oferuje licencji rodziny o pojemności 6 ani nieograniczonych planów grupowych.

\emph{TODO: modele testowe będą analizować, w jakim stopniu ograniczenie do małych grup (2--4 osoby) wpływa na strukturę rozwiązania i koszt w porównaniu z wariantem \texttt{duolingo\_super}.}

\section{Metodyka eksperymentów}

Testy z rozszerzonymi konfiguracjami zaplanowano zarówno w wariancie statycznym (benchmark), jak i dynamicznym:

\begin{itemize}
\item \textbf{Benchmark (statyczny)} -- dla rozmiarów $n = \{20, 50, 100, 200\}$, po 2 próbki na rozmiar i 1 powtórzenie. Algorytmy to ten sam zestaw co w rozdziale \ref{chap:experiments} (\textbf{ILPSolver}, \textbf{GreedyAlgorithm}, \textbf{RandomizedAlgorithm}, \textbf{DominatingSetAlgorithm}, \textbf{AntColonyOptimization}, \textbf{SimulatedAnnealing}, \textbf{TabuSearch}, \textbf{GeneticAlgorithm}). Czas obliczeń ograniczono do 45 s. Rejestrowane metryki: koszt całkowity, czas wykonania, koszt na węzeł, liczba grup i ich rozkład.

\item \textbf{Dynamiczne rozszerzenia} -- symulacje dynamiczne z rozmiarami $n = \{40, 80, 160\}$, liczbą kroków 15 i pojedynczym powtórzeniem. Mutacje odbywają się z parametrami identycznymi jak w rozdziale \ref{chap:dynamic} (niski poziom intensywności). Dla wariantów dynamicznych planowane są wykresy ewolucji kosztu i czasu w funkcji kroku oraz analiza wpływu warm startu na metaheurystyki.
\end{itemize}

\section{Prognozy i oczekiwane efekty}

Choć wyniki nie są jeszcze dostępne, na podstawie dotychczasowych obserwacji można sformułować kilka prognoz:

\begin{itemize}
\item \textbf{Wpływ parametru $p$ w roman\_p\_x} -- wzrost $p$ powinien prowadzić do zmniejszenia udziału licencji grupowej, a tym samym do wzrostu kosztu na węzeł. Iloraz $p/2$ wyznacza granicę opłacalności korzystania z licencji Solo versus Group.

\item \textbf{Duo w Spotify} -- obecność planu Duo (pojemność 2) prawdopodobnie obniży koszt w porównaniu z wersją \texttt{duolingo\_super} (która ma tylko pojemność 1 i 6). Algorytmy będą częściej tworzyć pary niż większe grupy, co może wpływać na czas (mniej złożone decyzje) oraz liczbę licencji Family w rozwiązaniu.

\item \textbf{Netflix} -- w konfiguracji Netflix spodziewany jest większy rozrzut wyników; ograniczenie Premium do 4 osób i brak licencji z nieograniczoną pojemnością może utrudnić pokrycie w bardzo gęstych grafach. Różnice między planami Standard i Premium mogą zależeć od wielkości grup.

\item \textbf{Algorytm ILPSolver} -- dla nowych konfiguracji kosztowych spodziewane jest, że ILP utrzyma przewagę jakościową dla małych grafów (jak w rozdziale \ref{chap:experiments}), ale będzie jeszcze bardziej ograniczony czasowo, ponieważ nowe warianty licencji zwiększą liczbę możliwych kombinacji.
\end{itemize}

\section{Planowane elementy graficzne i tabele (placeholdery)}

Do finalnej wersji rozdziału dodane zostaną następujące elementy:

\begin{itemize}
\item wykresy pudełkowe kosztu i czasu dla każdej konfiguracji (\texttt{roman\_p\_1\_5}, \texttt{roman\_p\_2\_5}, \texttt{roman\_p\_3\_0}, \texttt{spotify}, \texttt{netflix}),
\item linie trendu kosztu i czasu w funkcji liczby węzłów (dla statycznych eksperymentów),
\item fronty Pareto (koszt vs czas) dla poszczególnych konfiguracji,
\item wykresy pokazujące udział licencji Individual/Duo/Family w rozwiązaniu (dla Spotify i Netflix),
\item w wariancie dynamicznym: koszt i czas vs krok symulacji, podobnie jak w rozdziale \ref{chap:dynamic}.
\end{itemize}

\emph{TODO: rysunki i tabele zostaną wstawione po zakończeniu obliczeń.}
