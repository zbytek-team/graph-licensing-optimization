\chapter{Rozszerzenia modelu i dodatkowe konfiguracje}\label{chap:extensions}

Przyjęty w poprzednich rozdziałach model zakładał dwa rodzaje licencji (\texttt{duolingo\_super} i \texttt{roman\_domination}). W praktyce istnieje wiele wariantów subskrypcji o różnych cenach i limitach grupowych. Niniejszy rozdział opisuje rozszerzenia modelu uwzględniające:
\begin{itemize}
  \item różne poziomy ceny licencji grupowej w wariancie \texttt{roman\_domination} (parametr $p$),
  \item dodatkowy plan Duo występujący w ofercie serwisu Spotify,
  \item trzy warianty subskrypcji platformy Netflix.
\end{itemize}

Eksperymenty obejmujące te warianty zostały zaplanowane, lecz na moment powstawania rozdziału nie są jeszcze dostępne pliki wynikowe z przebiegu testów. Dlatego część sekcji dotyczących analizy i wyników oznaczono jako TODO.

\section{Warianty roman\_p\_x}

W bazowej konfiguracji \texttt{roman\_domination} występują dwa typy licencji:
\begin{itemize}
  \item \textbf{Solo} -- koszt 1,0 jednostki za jedną osobę,
  \item \textbf{Group} -- koszt 2,0 jednostki niezależnie od liczby osób (nieograniczona pojemność).
\end{itemize}

Warianty \texttt{roman\_p\_x} wprowadzają parametr $p > 1$, określający koszt licencji grupowej. W kodzie systemu wartość $p$ jest definiowana w nazwie konfiguracji: \texttt{roman\_p\_1\_5}, \texttt{roman\_p\_2\_5}, \texttt{roman\_p\_3\_0} odpowiadają odpowiednio kosztom 1,5; 2,5; 3,0 dla licencji grupowej. Pozostałe parametry (nielimitowana pojemność, koszt licencji Solo = 1,0) pozostają niezmienione.

Wzrost parametru $p$ powoduje większą relatywną opłacalność licencji Solo; w związku z tym oczekuje się, że dla wyższych $p$ algorytmy będą tworzyć mniejsze grupy, a w skrajnym przypadku (duży $p$) możliwy jest powrót do wariantu z dominacją klasyczną. Planowane analizy obejmują:
\begin{itemize}
  \item zależność kosztu na wierzchołek od wielkości grafu dla poszczególnych $p$,
  \item porównanie rozkładu wielkości grup przy różnych wartościach $p$,
  \item badanie udziału licencji Solo vs. Group w rozwiązaniach.
\end{itemize}

\emph{TODO: rysunki, tabele.}

\section{Konfiguracja Spotify}

Jak pokazano wcześniej w Tabeli~\ref{tab:license_models_real}, oprócz wariantu bazowego z jedną licencją indywidualną i jedną grupową istnieją w praktyce konfiguracje z dodatkowymi planami pośrednimi. Dobrym przykładem jest Spotify, gdzie występują trzy rodzaje licencji:
\begin{itemize}
  \item \textbf{Individual} -- cena 23,99 (jedna osoba, pojemność 1),
  \item \textbf{Duo} -- cena 30,99 (dokładnie 2 użytkowników, pojemność 2),
  \item \textbf{Family} -- cena 37,99 (od 2 do 6 osób).
\end{itemize}

W planie Duo dwóch abonentów dzieli wspólny plan, zachowując oddzielne konta premium. Wymagane jest, aby obie osoby mieszkały pod tym samym adresem. Dostępność licencji Duo wprowadza nową strategię grupowania: optymalny algorytm może preferować tworzenie par (pojemność 2) zamiast większych grup (pojemność 6).

\emph{TODO: porównanie koszt, czas, wykresy, front Pareto.}

\section{Konfiguracja Netflix}

Subskrypcja platformy Netflix została uproszczona do trzech planów \cite{netflix_plans}:
\begin{itemize}
  \item \textbf{Basic} -- cena 33,00 (dla 1 użytkownika),
  \item \textbf{Standard} -- cena 49,00 (dla 1--2 użytkowników, pojemność 2),
  \item \textbf{Premium} -- cena 67,00 (dla 1--4 użytkowników, pojemność 4).
\end{itemize}

W odróżnieniu od Spotify, Netflix nie oferuje licencji rodziny o pojemności 6 ani nieograniczonych planów grupowych; ponadto dodatkowi użytkownicy są dostępni jedynie jako płatne "miejsca" dla wyższych planów \cite{netflix_plans}.

\emph{TODO: wpływ małe grupy (2--4) na koszt i strukturę, porównanie z \texttt{duolingo\_super}.}

\section{Metodyka eksperymentów}

\end{itemize}


